%%%%%%%%%%%%%%%%%%%%%%%%%%%%%%%%%%%%%%%%%%%%%%%%

% Specify the command that you want into the header of the
% index.md file

%%%%%%%%%%%%%%%%%%%%%%%%%%%%%%%%%%%%%%%%%%%%%%%%

% Options for packages loaded elsewhere
\PassOptionsToPackage{unicode}{hyperref}
\PassOptionsToPackage{hyphens}{url}
\PassOptionsToPackage{dvipsnames,svgnames*,x11names*}{xcolor}
%
\documentclass[
  12pt,
  oneside]{report}
%%\usepackage{lmodern}
%
% Set line spacing
\usepackage{setspace}
\setstretch{1.5}

\usepackage{amssymb,amsmath}
\usepackage{ifxetex,ifluatex}
\ifnum 0\ifxetex 1\fi\ifluatex 1\fi=0 % if pdftex
  \usepackage[T1]{fontenc}
  \usepackage[utf8]{inputenc}
  \usepackage{textcomp} % provide euro and other symbols
\else % if luatex or xetex
  \usepackage{unicode-math}
  \defaultfontfeatures{Scale=MatchLowercase}
  \defaultfontfeatures[\rmfamily]{Ligatures=TeX,Scale=1}
\fi
% Use upquote if available, for straight quotes in verbatim environments
\IfFileExists{upquote.sty}{\usepackage{upquote}}{}
\IfFileExists{microtype.sty}{% use microtype if available
  \usepackage[]{microtype}
  \UseMicrotypeSet[protrusion]{basicmath} % disable protrusion for tt fonts
}{}
\makeatletter
\@ifundefined{KOMAClassName}{% if non-KOMA class
  \IfFileExists{parskip.sty}{%
    \usepackage{parskip}
  }{% else
    \setlength{\parindent}{0pt}
    \setlength{\parskip}{6pt plus 2pt minus 1pt}}
}{% if KOMA class
  \KOMAoptions{parskip=half}}
\makeatother
\usepackage{xcolor}
\IfFileExists{xurl.sty}{\usepackage{xurl}}{} % add URL line breaks if available
\IfFileExists{bookmark.sty}{\usepackage{bookmark}}{\usepackage{hyperref}}
\hypersetup{
  pdfauthor={François Leroy, PhD student at CZU},
  colorlinks=true,
  linkcolor=Blue,
  filecolor=Blue,
  citecolor=Blue,
  urlcolor=Blue,
  pdfcreator={LaTeX via pandoc}}
\urlstyle{same} % disable monospaced font for URLs

%% Package geometry
\usepackage[left = 2cm,right = 2cm,top = 2cm,bottom = 2cm]{geometry}
\usepackage{pdflscape}


%\usepackage{longtable} % out of date, now in latex-tools package
\usepackage{booktabs}
% Correct order of tables after \paragraph or \subparagraph
\usepackage{etoolbox}
\makeatletter
\patchcmd\longtable{\par}{\if@noskipsec\mbox{}\fi\par}{}{}
\makeatother
% Allow footnotes in longtable head/foot
\IfFileExists{footnotehyper.sty}{\usepackage{footnotehyper}}{\usepackage{footnote}}
\makesavenoteenv{longtable}
\usepackage{graphicx}
\makeatletter
\def\maxwidth{\ifdim\Gin@nat@width>\linewidth\linewidth\else\Gin@nat@width\fi}
\def\maxheight{\ifdim\Gin@nat@height>\textheight\textheight\else\Gin@nat@height\fi}
\makeatother
% Scale images if necessary, so that they will not overflow the page
% margins by default, and it is still possible to overwrite the defaults
% using explicit options in \includegraphics[width, height, ...]{}
\setkeys{Gin}{width=\maxwidth,height=\maxheight,keepaspectratio}
% Set default figure placement to htbp
\makeatletter
\def\fps@figure{htbp}
\makeatother
\setlength{\emergencystretch}{3em} % prevent overfull lines
\providecommand{\tightlist}{%
  \setlength{\itemsep}{0pt}\setlength{\parskip}{0pt}}
\setcounter{secnumdepth}{5}
%%% Complete the preamble of the LaTeX template
%%%------------------------------------------------------------------------------

%% Bug de bookdown: ne traite plus la déclaration "otherlangs" dans le préambule
% Pour charger les langues, écriture ici en dur du produit de bookdown
% Corrigé le 22/11/2019. A retester régulièrement: supprimer ces lignes si la compilation fonctionne sans elles.
\usepackage{polyglossia}
  \setmainlanguage[variant=american]{english}
  \setotherlanguage[]{french}
% Bug persistant le 28/02/2020

% Advised with polyglossia and babel
\usepackage{csquotes}

% Environnement "Essentiel" en début de chapitre
\usepackage[tikz]{bclogo}
\newenvironment{Essentiel}
  {\begin{bclogo}[logo=\bctrombone, noborder=true, couleur=lightgray!50]{L'essentiel}\parindent0pt}
  {\end{bclogo}}

%% Package fontspec
\usepackage{fontspec}
\setmainfont{calibri}[
  Path           = ./fonts/,
  Extension      = .ttf,
  BoldFont       = calibrib,
  ItalicFont     = calibrili,
  BoldItalicFont = calibriz]

% Rename chapters
% Below, scrpit to prevent the "chapter n" and the space use for it to
% be displayed
\usepackage{titlesec}
\titleformat{\chapter}   
{\Huge}{\thechapter{. }}{0pt}{\Huge}
%{\thechapter{. }}
\titlespacing*{\chapter}{0pt}{0pt}{10pt}
% empty space is to up the title and 10 is the space with the text below


% When using the natbib biblio package, includes "References" in the table of contents
\usepackage[nottoc]{tocbibind}

% To make the table caption full page width
\usepackage{caption}

% Make the figure float
\usepackage{float}

% In order to make the new chapter to start on the same page
\titleclass{\chapter}{straight}
\usepackage{float}
\usepackage{booktabs}
\usepackage{longtable}
\usepackage{array}
\usepackage{multirow}
\usepackage{wrapfig}
\usepackage{colortbl}
\usepackage{pdflscape}
\usepackage{tabu}
\usepackage{threeparttable}
\usepackage{threeparttablex}
\usepackage[normalem]{ulem}
\usepackage{makecell}
\usepackage{xcolor}
\ifluatex
  \usepackage{selnolig}  % disable illegal ligatures
\fi
\usepackage[style=apa,]{biblatex}
\addbibresource{references.bib}

\author{François Leroy, PhD student at CZU}
\date{2021-09-23}

% to include pdf
\usepackage{pdfpages}



%%%%%%%%%%%%%%%%%%%%%%%%%%%%%%%%%%%%%%%%%%%%%%%%%%%%%%%%%%%%%
% Start of the documents
\begin{document}

\includepdf[pages = {1}, fitpaper=true]{_assets/coverpage.pdf}

% Roman numbering for content before toc and toc itself
\cleardoublepage 
\pagenumbering{roman}

{
\hypersetup{linkcolor=}
\setcounter{tocdepth}{1}
\tableofcontents
\newpage
}
\vspace{50mm}
\setstretch{1.5}


% Start the arabic numbering at the 1st chapter
\cleardoublepage 
\pagenumbering{arabic}


% The mind, the...
\hypertarget{introduction}{%
\chapter{Introduction}\label{introduction}}

Human life quality is intrinsically linked to ecosystems state that he is living in. Indeed, ecosystems services extend in a large spectrum of mechanisms including nutrient cycle, food production, or climate and water cycle regulation \autocite{pereira_global_2012}. Some of those ecosystem functions are managed by bird biodiversity such as seed dispersal, controls pests or pollinate plant. Unfortunately, anthropogenic stressors like habitat loss, over exploitation, pollution or introduction of invasive species could lead biodiversity to its sixth mass extinction \autocite{barnosky_has_2011}.

We now know that the loss of global biodiversity is unprecedented and political decisions has been stated in order to limit it \autocite[\emph{e.g.}][2010, 2002]{secretariat_of_the_convention_on_biological_diversity_global_2006}. However, current scientific literature has also shown that temporal trends in local changes of biodiversity can be opposite to trends at larger scales \autocite[\emph{e.g.}][]{chase_species_2019}. Thus, current changes in biodiversity is far more complex than a simple global decrease: most of the ecosystems undergo alterations of their communities with changes in species composition \autocite{blowes_geography_2019,dornelas_quantifying_2013,vaidyanathan_worlds_2021}.

Since \textcite{arrhenius_species_1921} and \textcite{grinnell_role_1922}, we know that spatial and temporal scaling of biodiversity can drive macroecological patterns. If the scaling of biodiversity has been of great interest these last decades \autocite[\emph{e.g.}][]{storch_scaling_2007}, wonders persist about how the trends of biodiversity are linked to the spatial and temporal scales. In other words: why and how is the scaling of biodiversity trends important?

Here, the term ``spatial grain'' is also used to refer to the spatial scale of biodiversity, \emph{i.e.} the area at which the metric of biodiversity is assessed. One should be careful to not confuse spatial grain with the spatial extent of a study, \emph{i.e.} the total area of the ecosystem studied. The same terminology is applied for the temporal scale: temporal grain refers to the temporal unit of the measured biodiversity. This review will, in part, show how this definition has still no consensus in the scientific literature.

In order to investigate this link between spatio-temporal grains and trends of biodiversity, birds is a relevant taxon. As a matter of fact, thanks to the many ornithological monitoring and surveys, we now have a large number of long, structured time series on bird populations \autocites{bejcek_velke_2016,sauer_north_2013,kamp_population_2021}. Birds are easy to observe, easy to identify and thus many volunteers are motivated to conduct standardized sampling. Given their ability to change quickly of locations, their presence is also a good indicator for ecosystem health and thus several standardized metrics have been created to assess their populations.

Studying biodiversity can be confusing, especially because several choices must be done. Firstly, the level at which you are looking at the biodiversity must be chosen (\emph{e.g.} taxonomic, functional, phylogenetic diversity). Secondly, one must decide which metric is the most appropriate for his study. There are many facets of biodiversity that can be measured by different metrics depending on the objective of the study. Measures of static biodiversity are commonly used such as species richness or \(\alpha\) diversity \autocite[\emph{i.e.} number of species,][]{whittaker_vegetation_1960}, the Shannon index \autocite{shannon_mathematical_1948} ,the Simpson index \autocite{simpson_measurement_1949} or the Hill number \autocite{hill_diversity_1973}. The later three biodiversity indexes take into account the relative abundances of the species and thus can be considered as \emph{qualitative indexes} of biodiversity. On an other hand, the spatial and temporal \(\beta\) diversity will measure the species turnover and can be measured thanks to Whittaker's \autocite{whittaker_evolution_1972}, Sørensen's \autocite{sorensen_method_1948} or Jaccard's \autocite{jaccard_distribution_1912} dissimilarity indexes \autocite[\emph{e.g.}][]{keil_patterns_2012}. All these metrics assess species-based metrics, \emph{i.e.} they use the species as a unit. However, it has also been shown that functional and phylogenetic diversity can provide supplementary information on the community structure and its dynamic \autocites[\emph{e.g.}][]{mcgill_rebuilding_2006,mouquet_ecophylogenetics_2012,webb_phylogenies_2002}.

Other metrics of great interest are the abundance-based or population-based metrics. As individuals react to disturbances of ecosystems by disappearing, the population trends are considered as resulting of the ecosystems health. As a matter of fact, before going locally extinct, populations go through a lot of steps that can not be considered by species-based metric. Thus, population trends are usually efficient at assessing biodiversity declines. Although overall populations are impossible to assess, determining the abundance of few indicator species can give a relevant summary on how is going the entire ecosystem \autocite{gregory_developing_2005}. These family of metrics are called the multi-species indicators (MSI) and we have seen the emergence of several. We can cite here the farmland bird indicator, woodland bird indicator or one of the most informative, which summarizes these two previous metrics: the Wildland Bird Indicator \autocite{gregory_generation_1999,gregory_wild_2010}. The main idea of these metrics is to compute the geometric mean of abundance of few key species over time.

Finally a last class of indicators, not considered here, take into account both species and ecosystems feature into a summarizing index: the composite indicators. The most known ones are the Red List Index \autocite{butchart_improvements_2007,butchart_using_2005,butchart_measuring_2004} or the Biodiversity Change Index \autocite{normander_indicator_2012}.

Here, I propose to review articles assessing the temporal trends of avian biodiversity metrics and to look at which spatial and temporal grains these trends have been assessed. I decided to consider the most common macroecological indicators used to assess biodiversity at the community level and higher, such as diversity indexes (\emph{e.g.} species richness, functional diversity\ldots) or population indexes \autocite{mcgill_fifteen_2015}. Summarizing the trends of these qualitative and/or quantitative avian biodiversity indexes along with their spatial and temporal grains will help to see more clearly how scaling trends of biodiversity is important. Lack of consensus about specifications and definitions of both spatial and temporal grains (respectively) of trends is also highlighted here. Doubtlessly, it is valuable to demonstrate that scaling trends is seldom considered as it can bring confusion to the analysis and comparisons of these trends. Moreover, I show that literature which uses spatial replicates to make the trend computed robust is not the standard. I believe that this review can help to have a better overview of the current knowledge on the spatio-temporal scaling of trends biodiversity, using bird as a taxa of reference.

\hypertarget{materials-and-methods}{%
\chapter{Materials and Methods}\label{materials-and-methods}}

For this review, articles of interest were the ones assessing temporal trends of the most common indicators of avian biodiversity and specifying spatial and temporal scales. I decided to take into account only the articles for which there were spatial replicates, \emph{i.e.} where the trend of the metric was assessed at several locations with the same spatial grain. With these replications, the trend reported at one spatial grain is more reliable. However, at larger spatial grains (\emph{i.e.} national, continental or global scales), spatial replicates are rarer. Thus, I decided to take note of these unreplicated trends only when they were found in a article were spatial replicates were used at smaller scales.

I used the \emph{``advanced search''} tool of the ISI Web of Science Core collection database with these four following queries:

\begin{enumerate}
\def\labelenumi{\arabic{enumi}.}
\item
  \texttt{AB\ =\ ((biodiversity\ OR\ species\ richness\ OR\ diversity)\ AND\ (temporal\ trend*\ OR\ dynamic*)\ AND\ (bird*\ OR\ avia*))} which resulted in 1346 references.
\item
  \texttt{AB\ =\ ((biodiversity\ change\ index)\ \ AND\ (bird*\ \ OR\ avia*)\ \ AND\ trend*)} which resulted in 60 references.
\item
  \texttt{AB\ =\ ((species\ richness)\ AND\ (bird*\ OR\ avia*)\ AND\ trend*)} which resulted in 313 references.
\item
  \texttt{ALL=(birds\ AND\ species\ richness\ AND\ temporal\ trend)} which resulted in 88 references.
\end{enumerate}

For each query, the title and abstract of the articles were reviewed. When the temporal trend was explicitly specified (either visually or literally), the material and method part was read in order to collect: the type of \emph{metric}, the \emph{spatial grain} of the trend (\emph{i.e.} the area at which the metric trend is assessed), its \emph{temporal grain}, the \emph{spatial extent} (\emph{i.e.} the entire area on which the study applies), the \emph{temporal extent} and the \emph{beginning and ending years} of the study as well as the \emph{general trend} of the metric (Table \ref{tab:maintable}). Alternatively, the articles which were often referred to in the relevant articles were also explored.

Spatial grain sizes are discretized into four levels: \emph{local} \(<= 25\) \(Km^2\), \emph{regional} \(> 25\) \(Km²\), \emph{national} when an entire country is considered and \emph{global} at the worldwide scale.

Concerning the trend assessment, different papers contain the \emph{p-value} or directly specify the significant trend of the metric. However, a portion of papers gives only visual representations of the trend. For those, the standard error was used when displayed. For the very few only giving the trend, \textcolor{red}{the rule of thumb was applied}. Information can be found in the column \emph{Note} of the Table \ref{tab:notetable}. Moreover, the final trend retained (\emph{i.e.} either \emph{Increase}, \emph{Stable} or \emph{Decrease}) doesn't reflect all the fluctuations of the metric through time but rather the difference between the starting and ending points.

I have found 31 references in which authors were both determining the temporal trend of a metric and explicitly defining the grain size. However, only 17 of them are using spatial replicates and are thus relevant for this study (Table \ref{tab:maintable}). The classes of metric are: \emph{Species richness, Evenness, Abundance, Diversity, Temporal beta-diversity, Spatial beta-diversity, Functional diversity, Functional evenness, Functional richness}. Some of this classes contain several different indexes. This is the case for instance for the class \emph{Diversity}, which can be defined by either the Shannon or the Simpson index, or for the class \emph{Abundance} which contains various multi-species indicators (for more information, see Table \ref{tab:notetable}).

\begin{landscape}\begingroup\fontsize{10}{12}\selectfont

\begin{longtable}[t]{>{\raggedright\arraybackslash}p{6.5em}>{\raggedright\arraybackslash}p{6.5em}>{\raggedright\arraybackslash}p{6.5em}>{\raggedleft\arraybackslash}p{6.5em}>{\raggedleft\arraybackslash}p{6.5em}>{\raggedleft\arraybackslash}p{6.5em}>{\raggedright\arraybackslash}p{6.5em}>{\raggedright\arraybackslash}p{6.5em}>{\raggedright\arraybackslash}p{6.5em}}
\caption{\label{tab:maintable}Trends of different metrics of biodiversity at various spatial and temporal scales}\\
\toprule
Reference & Metric & Spatial grain (Km²) & Temporal grain (year) & Spatial extent (Km²) & Temporal extent (year) & Years & Country & Trend\\
\midrule
\endfirsthead
\caption[]{\label{tab:maintable}Trends of different metrics of biodiversity at various spatial and temporal scales \textit{(continued)}}\\
\toprule
Reference & Metric & Spatial grain (Km²) & Temporal grain (year) & Spatial extent (Km²) & Temporal extent (year) & Years & Country & Trend\\
\midrule
\endhead

\endfoot
\bottomrule
\endlastfoot
\cellcolor{gray!6}{\cite{barnagaud_temporal_2017}} & \cellcolor{gray!6}{Evenness} & \cellcolor{gray!6}{Local} & \cellcolor{gray!6}{1.0} & \cellcolor{gray!6}{9834000} & \cellcolor{gray!6}{41} & \cellcolor{gray!6}{1970-2011} & \cellcolor{gray!6}{USA} & \cellcolor{gray!6}{Increase}\\
 & SR & Local & 1.0 & 9834000 & 41 & 1970-2011 & USA & Increase\\
\cellcolor{gray!6}{\cite{bowler_geographic_2021}} & \cellcolor{gray!6}{Abundance} & \cellcolor{gray!6}{National} & \cellcolor{gray!6}{1.0} & \cellcolor{gray!6}{520475} & \cellcolor{gray!6}{27} & \cellcolor{gray!6}{1990-2016} & \cellcolor{gray!6}{Czech Rep., Switzerland, Denmark, Germany} & \cellcolor{gray!6}{Stable}\\
\cite{chase_species_2019} & SR & Local & 5.0 & 2800000 & 30 & 1982–2011 & USA, Canada & Stable\\
\cellcolor{gray!6}{} & \cellcolor{gray!6}{SR} & \cellcolor{gray!6}{Regional} & \cellcolor{gray!6}{5.0} & \cellcolor{gray!6}{2800000} & \cellcolor{gray!6}{30} & \cellcolor{gray!6}{1982–2011} & \cellcolor{gray!6}{USA, Canada} & \cellcolor{gray!6}{\vphantom{1} Stable}\\
\addlinespace
 & SR & Regional & 5.0 & 2800000 & 30 & 1982–2011 & USA, Canada & Stable\\
\cellcolor{gray!6}{} & \cellcolor{gray!6}{SR} & \cellcolor{gray!6}{Local} & \cellcolor{gray!6}{5.0} & \cellcolor{gray!6}{2800000} & \cellcolor{gray!6}{30} & \cellcolor{gray!6}{1982–2011} & \cellcolor{gray!6}{USA, Canada} & \cellcolor{gray!6}{\vphantom{1} Increase}\\
 & SR & Local & 5.0 & 2800000 & 30 & 1982–2011 & USA, Canada & Increase\\
\cite{chiron_forecasting_2013} & Abundance & Regional & 1.0 & 643801 & 14 & 2007-2020 & France & \vphantom{2} Decrease\\
\cellcolor{gray!6}{} & \cellcolor{gray!6}{Abundance} & \cellcolor{gray!6}{Regional} & \cellcolor{gray!6}{1.0} & \cellcolor{gray!6}{643801} & \cellcolor{gray!6}{14} & \cellcolor{gray!6}{2007-2020} & \cellcolor{gray!6}{France} & \cellcolor{gray!6}{\vphantom{1} Decrease}\\
\addlinespace
 & Abundance & Regional & 1.0 & 643801 & 14 & 2007-2020 & France & Decrease\\
 & Abundance & Regional & 1.0 & 643801 & 14 & 2007-2020 & France & Decrease\\
\cellcolor{gray!6}{\cite{davey_rise_2012}} & \cellcolor{gray!6}{Diversity} & \cellcolor{gray!6}{Local} & \cellcolor{gray!6}{1.0} & \cellcolor{gray!6}{242495} & \cellcolor{gray!6}{13} & \cellcolor{gray!6}{1994-2006} & \cellcolor{gray!6}{UK} & \cellcolor{gray!6}{Increase}\\
 & Evenness & Local & 1.0 & 242495 & 13 & 1994-2006 & UK & Increase\\
\cellcolor{gray!6}{} & \cellcolor{gray!6}{SR} & \cellcolor{gray!6}{Local} & \cellcolor{gray!6}{1.0} & \cellcolor{gray!6}{242495} & \cellcolor{gray!6}{13} & \cellcolor{gray!6}{1994-2006} & \cellcolor{gray!6}{UK} & \cellcolor{gray!6}{Increase}\\
\addlinespace
\cite{harrison_assessing_2014} & Abundance & Local & 1.0 & 200000 & 18 & 1994-2011 & Great Britain, UK & Increase\\
\cellcolor{gray!6}{} & \cellcolor{gray!6}{Abundance} & \cellcolor{gray!6}{Local} & \cellcolor{gray!6}{1.0} & \cellcolor{gray!6}{200000} & \cellcolor{gray!6}{18} & \cellcolor{gray!6}{1994-2011} & \cellcolor{gray!6}{Great Britain, UK} & \cellcolor{gray!6}{\vphantom{1} Stable}\\
 & Abundance & Local & 1.0 & 200000 & 18 & 1994-2011 & Great Britain, UK & Stable\\
\cellcolor{gray!6}{\cite{harrison_quantifying_2016}} & \cellcolor{gray!6}{Abundance} & \cellcolor{gray!6}{Local} & \cellcolor{gray!6}{0.5} & \cellcolor{gray!6}{242495} & \cellcolor{gray!6}{20} & \cellcolor{gray!6}{1994-2013} & \cellcolor{gray!6}{UK} & \cellcolor{gray!6}{Increase}\\
 & Abundance & Local & 0.5 & 242495 & 20 & 1994-2013 & UK & \vphantom{1} Stable\\
\addlinespace
\cellcolor{gray!6}{} & \cellcolor{gray!6}{Abundance} & \cellcolor{gray!6}{Local} & \cellcolor{gray!6}{0.5} & \cellcolor{gray!6}{242495} & \cellcolor{gray!6}{20} & \cellcolor{gray!6}{1994-2013} & \cellcolor{gray!6}{UK} & \cellcolor{gray!6}{Stable}\\
\cite{jarzyna_taxonomic_2018}\cellcolor{gray!6}{} & \cellcolor{gray!6}{SR} & \cellcolor{gray!6}{Regional} & \cellcolor{gray!6}{1.0} & \cellcolor{gray!6}{9834000} & \cellcolor{gray!6}{45} & \cellcolor{gray!6}{1969-2013} & \cellcolor{gray!6}{USA} & \cellcolor{gray!6}{\vphantom{1} Increase}\\
 & SR & Regional & 1.0 & 9834000 & 45 & 1969-2013 & USA & Increase\\
 & SR & Regional & 1.0 & 9834000 & 45 & 1969-2013 & USA & Increase\\
\cellcolor{gray!6}{} & \cellcolor{gray!6}{SR} & \cellcolor{gray!6}{National} & \cellcolor{gray!6}{1.0} & \cellcolor{gray!6}{9834000} & \cellcolor{gray!6}{45} & \cellcolor{gray!6}{1969-2013} & \cellcolor{gray!6}{USA} & \cellcolor{gray!6}{Increase}\\
\addlinespace
 & SR & Global & 1.0 & 148940000 & 45 & 1969-2013 & World & Decrease\\
\cellcolor{gray!6}{} & \cellcolor{gray!6}{Temporal beta-diversity} & \cellcolor{gray!6}{Regional} & \cellcolor{gray!6}{1.0} & \cellcolor{gray!6}{9834000} & \cellcolor{gray!6}{45} & \cellcolor{gray!6}{1969-2013} & \cellcolor{gray!6}{USA} & \cellcolor{gray!6}{\vphantom{2} Increase}\\
 & Temporal beta-diversity & Regional & 1.0 & 9834000 & 45 & 1969-2013 & USA & \vphantom{1} Increase\\
\cellcolor{gray!6}{} & \cellcolor{gray!6}{Temporal beta-diversity} & \cellcolor{gray!6}{Regional} & \cellcolor{gray!6}{1.0} & \cellcolor{gray!6}{9834000} & \cellcolor{gray!6}{45} & \cellcolor{gray!6}{1969-2013} & \cellcolor{gray!6}{USA} & \cellcolor{gray!6}{Increase}\\
 & Temporal beta-diversity & National & 1.0 & 9834000 & 45 & 1969-2013 & USA & Increase\\
\addlinespace
\cellcolor{gray!6}{} & \cellcolor{gray!6}{Temporal beta-diversity} & \cellcolor{gray!6}{Global} & \cellcolor{gray!6}{1.0} & \cellcolor{gray!6}{148940000} & \cellcolor{gray!6}{45} & \cellcolor{gray!6}{1969-2013} & \cellcolor{gray!6}{World} & \cellcolor{gray!6}{Stable}\\
\cite{pilotto_meta-analysis_2020} & Abundance & Local & NA & 10180000 & NA & NA & Europe & Stable\\
\cellcolor{gray!6}{} & \cellcolor{gray!6}{Diversity} & \cellcolor{gray!6}{Local} & \cellcolor{gray!6}{NA} & \cellcolor{gray!6}{10180000} & \cellcolor{gray!6}{NA} & \cellcolor{gray!6}{NA} & \cellcolor{gray!6}{Europe} & \cellcolor{gray!6}{Increase}\\
 & SR & Local & NA & 10180000 & NA & NA & Europe & Increase\\
\cellcolor{gray!6}{} & \cellcolor{gray!6}{Temporal beta-diversity} & \cellcolor{gray!6}{Local} & \cellcolor{gray!6}{NA} & \cellcolor{gray!6}{10180000} & \cellcolor{gray!6}{NA} & \cellcolor{gray!6}{NA} & \cellcolor{gray!6}{Europe} & \cellcolor{gray!6}{Stable}\\
\addlinespace
\cite{ram_what_2017} & Abundance & Local & 1.0 & 350000 & 18 & 1998-2015 & Sweden & Increase\\
\cellcolor{gray!6}{} & \cellcolor{gray!6}{SR} & \cellcolor{gray!6}{Regional} & \cellcolor{gray!6}{1.0} & \cellcolor{gray!6}{350000} & \cellcolor{gray!6}{18} & \cellcolor{gray!6}{1998-2015} & \cellcolor{gray!6}{Sweden} & \cellcolor{gray!6}{Increase}\\
\cite{reif_changes_2013} & Spatial beta-diversity & Local & 1.0 & 79000 & 23 & 1982-2004 & Czech Rep. & Stable\\
\cellcolor{gray!6}{} & \cellcolor{gray!6}{SR} & \cellcolor{gray!6}{Local} & \cellcolor{gray!6}{1.0} & \cellcolor{gray!6}{79000} & \cellcolor{gray!6}{23} & \cellcolor{gray!6}{1982-2004} & \cellcolor{gray!6}{Czech Rep.} & \cellcolor{gray!6}{Stable}\\
 & SR & National & 1.0 & 79000 & 23 & 1982-2004 & Czech Rep. & Stable\\
\addlinespace
\cellcolor{gray!6}{\cite{schipper_contrasting_2016}} & \cellcolor{gray!6}{Abundance} & \cellcolor{gray!6}{Local} & \cellcolor{gray!6}{5.0} & \cellcolor{gray!6}{24710000} & \cellcolor{gray!6}{40} & \cellcolor{gray!6}{1971-2010} & \cellcolor{gray!6}{Canada, USA, Mexico} & \cellcolor{gray!6}{Increase}\\
 & Diversity & Local & 5.0 & 24710000 & 40 & 1971-2010 & Canada, USA, Mexico & \vphantom{1} Increase\\
\cellcolor{gray!6}{} & \cellcolor{gray!6}{Diversity} & \cellcolor{gray!6}{Local} & \cellcolor{gray!6}{5.0} & \cellcolor{gray!6}{24710000} & \cellcolor{gray!6}{40} & \cellcolor{gray!6}{1971-2010} & \cellcolor{gray!6}{Canada, USA, Mexico} & \cellcolor{gray!6}{Increase}\\
 & Functional diversity & Local & 5.0 & 24710000 & 40 & 1971-2010 & Canada, USA, Mexico & Decrease\\
\cellcolor{gray!6}{} & \cellcolor{gray!6}{Functional evenness} & \cellcolor{gray!6}{Local} & \cellcolor{gray!6}{5.0} & \cellcolor{gray!6}{24710000} & \cellcolor{gray!6}{40} & \cellcolor{gray!6}{1971-2010} & \cellcolor{gray!6}{Canada, USA, Mexico} & \cellcolor{gray!6}{Increase}\\
\addlinespace
 & Functional richness & Local & 5.0 & 24710000 & 40 & 1971-2010 & Canada, USA, Mexico & Increase\\
\cellcolor{gray!6}{} & \cellcolor{gray!6}{SR} & \cellcolor{gray!6}{Local} & \cellcolor{gray!6}{5.0} & \cellcolor{gray!6}{24710000} & \cellcolor{gray!6}{40} & \cellcolor{gray!6}{1971-2010} & \cellcolor{gray!6}{Canada, USA, Mexico} & \cellcolor{gray!6}{Increase}\\
\cite{sorte_changes_2005} & Abundance & Local & 1.0 & 9834000 & 36 & 1968-2003 & USA & Decrease\\
\cellcolor{gray!6}{} & \cellcolor{gray!6}{Evenness} & \cellcolor{gray!6}{Local} & \cellcolor{gray!6}{1.0} & \cellcolor{gray!6}{9834000} & \cellcolor{gray!6}{36} & \cellcolor{gray!6}{-} & \cellcolor{gray!6}{USA} & \cellcolor{gray!6}{Decrease}\\
 & SR & Local & 1.0 & 9834000 & 36 & 1968-2003 & USA & Increase\\
\addlinespace
\cellcolor{gray!6}{\cite{van_turnhout_scale-dependent_2007}} & \cellcolor{gray!6}{SR} & \cellcolor{gray!6}{Regional} & \cellcolor{gray!6}{4.0} & \cellcolor{gray!6}{41543} & \cellcolor{gray!6}{28} & \cellcolor{gray!6}{1973-2000} & \cellcolor{gray!6}{Netherlands} & \cellcolor{gray!6}{Increase}\\
 & SR & Local & 4.0 & 41543 & 28 & 1973-2000 & Netherlands & Increase\\
\cellcolor{gray!6}{} & \cellcolor{gray!6}{SR} & \cellcolor{gray!6}{National} & \cellcolor{gray!6}{4.0} & \cellcolor{gray!6}{41543} & \cellcolor{gray!6}{28} & \cellcolor{gray!6}{1973-2000} & \cellcolor{gray!6}{Netherlands} & \cellcolor{gray!6}{Increase}\\
\cite{wretenberg_changes_2010} & SR & Local & 1.0 & 1800 & 11 & 1994-2004 & Sweden & Decrease\\
\cellcolor{gray!6}{\cite{inger_common_2015}} & \cellcolor{gray!6}{Abundance} & \cellcolor{gray!6}{National} & \cellcolor{gray!6}{1.0} & \cellcolor{gray!6}{10180000} & \cellcolor{gray!6}{22} & \cellcolor{gray!6}{1980-2001} & \cellcolor{gray!6}{Europe} & \cellcolor{gray!6}{Decrease}\\
\addlinespace
\cite{donald_agricultural_2001} & Abundance & National & NA & 10180000 & 21 & 1970-1990 & Europe & Decrease\\*
\end{longtable}
\endgroup{}
\end{landscape}

\hypertarget{on-the-link-between-spatial-grain-and-trends-of-biodiversity}{%
\chapter{On the link between spatial grain and trends of biodiversity}\label{on-the-link-between-spatial-grain-and-trends-of-biodiversity}}

The median spatial extent of the 17 articles is \ensuremath{5.20475\times 10^{5}} \(Km^2\), with the smallest area of 1800 \(Km^2\) and the greatest representing the global emerged surface (\emph{i.e.} \ensuremath{1.4894\times 10^{8}} \(Km^2\)). Overall, there were 11 \emph{Decrease}, 31 \emph{Increase} and 14 \emph{Stable} reliable trends (\emph{i.e.} spatially replicated) across the literature, without consideration of any metric or spatial scales.

One have to keep in mind that the proportions of the trends presented here is not representative of the overall trend usually found in the scientific literature. Taking only the articles with spatial replicates reduces the number of studies and thus doesn't reflect the overall dynamic of biodiversity, which is not the purpose of this review. As a matter of fact, meta-analysis by \textcite{fraixedas_state_2020} indicates that around half of bird related indicators are declining and that only \(20\%\) of them are increasing.

In our case, local scales are more represented than the others and the number of articles decreases with the increasing spatial scale (Figure \ref{fig:barspatscale}). This is expected, as the spatial replications get more demanding in organization and resources as the grain size enlarges. The \emph{Increase} of the metrics seems to be dominating at smaller scales. On an other hand, the proportion of \emph{Decrease} is more important at regional scales than at local scales. At the global scale, no \emph{Increase} were found.

\begin{figure}
\centering
\includegraphics{literature_review_files/figure-latex/barspatscale-1.pdf}
\caption{\label{fig:barspatscale}Proportion of \emph{Increase}, \emph{Decrease} or \emph{Stable} trends for each spatial scale}
\end{figure}

Replicated species richness and abundance metrics are very represented in the scientific literature (Figure \ref{fig:barmetrics}). The less common trend of abundance is \emph{Increase}, whilst \emph{Decrease} and \emph{Stability} are both as common. Diversity indexes (\emph{i.e.} Sørensen and Jaccard) were always found increasing and temporal \(\beta\)-diversity was found most of the time increasing and never decreasing. The small frequency of the other metrics doesn't allow one to discuss the trend proportions.

\begin{figure}
\centering
\includegraphics{literature_review_files/figure-latex/barmetrics-1.pdf}
\caption{\label{fig:barmetrics}Proportion of \emph{Increase}, \emph{Decrease} or \emph{Stable} trends for each of the metric}
\end{figure}

\textbf{At local scales}, species richness experiences dominantly increase (Figure \ref{fig:barmetricsperspatscale}). Evenness indexes, \emph{i.e.} taxonomic and functional evenness, are mainly found increasing. Concerning the abundance indexes, the stability dominates but the increase is also often witnessed. In contrast, \textbf{at regional scale}, abundance metrics always decrease, whilst temporal \(\beta\)-diversity always increase. Species richness experiments mainly increases, then stability but no decrease is reported. Concerning \textbf{national and global scales}, one have to keep in mind that spatial replicates are less frequent and that most of the trends reported here for these two spatial scales are not replicated. Exceptions are made by \textcite{bowler_geographic_2021} and \textcite{donald_agricultural_2001}. The former showed negative trends in abundance indexes for Denmark and Germany and positive trends for Switzerland and Czech Republic, indicating that one can not conclude about general trend at national scale (here referred as \emph{Stable}). However, for \textcite{donald_agricultural_2001}, trends of mean populations were computed for 30 European countries and the majority was negative. No positive trend seem dominant at these spatial scales. Global trends only come from \textcite{jarzyna_taxonomic_2018}.

\begin{figure}
\centering
\includegraphics{literature_review_files/figure-latex/barmetricsperspatscale-1.pdf}
\caption{\label{fig:barmetricsperspatscale}Proportion of \emph{Increase}, \emph{Decrease} or \emph{Stable} trends for each metric. Each panel represent one spatial scale}
\end{figure}

Number of studies assessing trends with spatial replicates is clearly limited. Thus, assessing robust rules about spatial scaling of trends of biodiversity is not possible with the current state of the scientific literature. However, concerning species richness, we can see that the decrease is global, but that this decrease is rare at lower spatial scales. In fact, increases are more often observed, confirming the high perturbations that biodiversity is undergoing \autocite{dornelas_assemblage_2014,vaidyanathan_worlds_2021}. This analysis goes along with the temporal \(\beta\)-diversity which is always observed either stable or increasing, sign of the disturbances of the ecosystems.

On an other hand, the decrease of abundances seem to appear mainly at regional or national scales. \textcolor{red}{This may be due to the fact that local scale samplings are done usually outside of very anthropized  ecosystems, such as cities.Increasing the scale to regional and national increases the proportion of cities and farmland area, which can have negative impact on bird populations} \autocite{reif_long-term_2013}.

\hypertarget{a-lack-of-consistency}{%
\chapter{A lack of consistency}\label{a-lack-of-consistency}}

\hypertarget{spatial-replicates-are-not-a-standard}{%
\section{Spatial replicates are not a standard}\label{spatial-replicates-are-not-a-standard}}

Articles computing the trend of a metric with spatial replicates were limited, either due to a lack of data or because the trend was assessed for the spatial extent of the data. For instance, the Breeding Bird Surveys \autocites[\emph{e.g.}][]{sauer_north_2013,kamp_population_2021} usually follow standardized sampling plan with spatial replications (\emph{i.e.} census plots). However, not all the trend reported for the BBS are summarized at their specific grain sizes and were sometimes computed at their respective national scales, therefore without spatial replication. For instance, the common method encountered to assess population trends (\emph{i.e.} abundance indexes) is to learn a predictive model from the data, predict the target feature (\emph{i.e.} abundance) and then compute the metric and its trend from the output of the model at the national spatial extent \autocites[\emph{e.g.}][]{jiguet_modeling_2005,jiguet_french_2012,eglington_disentangling_2012,doxa_low-intensity_2010,sauer_first_2017}. These analysis are essential from a conservation point of view and are a majority \autocite{fraixedas_state_2020}: they give valuable information about ecosystem health at large spatial extent and are thus used as arguments by decision-makers. However, they represent just one trend amongst all the possible trends that can be found at one spatial scale and thus don't indicate robust relationships between spatial grain and temporal trend. An other common type of studies are the one using the space-for-time substitution \autocite{walker_use_2010} to assess the trend of a metric \autocite[one of the best example is][]{hill_determining_2004}. One could think that using this studies could increase significantly the spatial replicates. However, the space-for-time substitution is mainly used to assess the impact of a processes (\emph{e.g.} before/after logging, before/after urbanization etc) meaning that the trend computed is highly biased, which we try to avoid for our topic.

Even less articles were computing the trends of metrics with spatial replicates at various spatial grain. This was the case for only \textcite{chase_species_2019} and \textcite{jarzyna_taxonomic_2018}. It is also important to cite here \textcite{jarzyna_spatial_2015}, who did iterative computing of temporal change community metrics (\emph{i.e.} temporal dissimilarity, temporal turnover, extinction and colonization) at several spatial scales. However, the temporal trend of these metrics weren't considered and are therefore not reported in Table \ref{tab:maintable}.

\hypertarget{metric-heterogeneity}{%
\section{Metric heterogeneity}\label{metric-heterogeneity}}

Macroecology has seen its metrics blooming these last decades, in search of a multiparametric indexes taking into account all the components of an ecosystem, as do the composite indicators \autocites[\emph{e.g.} the red list index,][]{butchart_improvements_2007}[biodiversity change indexes,][]{normander_indicator_2012}. The latter have proven very effective in order to assess conservation policies and are able to synthesize the ecosystem complexities. This has been especially the case for the birds, who are of great interest in assessing the dynamic of ecosystem health \autocite{fraixedas_state_2020}. Thus, studies focusing only on macroecological metrics are less common than expected. Coupled with the need of spatial replicates, studying the link between the trend of \emph{classical} macroecological metrics and spatio-temporal scales still needs to be explored.

\hypertarget{on-the-temporal-scale}{%
\section{On the temporal scale}\label{on-the-temporal-scale}}

The oldest study started in 1968 and the median duration is 23 years, with a minimum time-span of 11 years and a maximum of 45.

Determination of the temporal grain was way more complicated, as there is no consensus on the subject. Usually, the temporal grain of the sampling is specified, but sometimes with inaccuracies (\emph{e.g.} ``in the early morning''). However, the temporal grain of the sampling plan doesn't represent the final temporal grain. For instance, some metrics are summed over a certain spatial extent \autocite[e.g.~summing the species richness over an atlas square, like in][]{van_turnhout_scale-dependent_2007}. In this case, the temporal grain should also be summed, but this was not specified. On an other hand, when the metric is averaged over a spatial grain, the sampling temporal grain can be considered.

Nevertheless, an other problem arises when the trend is computed. Usually, one compute the metric for a time \(t\) and considers the value to be representative of the time-span between \(t\) and \(t+1\) (\emph{e.g.} a day, a year\ldots). In this case, there is no consensus on what should be the temporal grain: should one keep the grain of the sampling? Should one consider it to be the time-span covered by the sampling? In my analysis, the latter has been used (Table \ref{tab:maintable}).

For the cases where the metric is determined out of model \autocite[e.g.][]{harrison_assessing_2014}, it is usually easier to assess the temporal grain. Indeed, predictions allow one to extrapolate the data from the sampling temporal grain to a wanted one. Thus, for these cases, the final temporal grain was most of the time explicitly given, with an order of length of the year.

In short, temporal grain has no clear definition as it is different according to the metric computed, how it is computed and if the temporal trend is assessed. Even though it is now clear that the temporal grain of the sampling is important, there is no such consensus regarding the temporal grain of the trend.

Highest temporal grains are found with increasing spatial extent (Figure \ref{fig:spacetimegrain}, \textcolor{red}{really not sure about this one}). It is due to the fact that data used in the selected article are mainly structured data, \emph{i.e.} data following a well established sampling plan. This type of survey need resources and organizations which makes them sparser. Increasing the total area to be sampled makes \emph{de facto} the time to increase and eventually the temporal grain. However, this entanglement between spatial extent and temporal grain is only related to the constraints of the sampling. This limitation can be overcome thanks to citizen science data, which are nowadays more and more used \autocites[\emph{e.g.}][]{bowler_winners_2021,isaac_data_2020,isaac_statistics_2014}. The opportunistic nature of these data make the number of observations more important for very short census times, even over a large area. These data, with high temporal grain resolution for large spatial scales, could be used to explore in more details the temporal scaling of biodiversity trend.

\begin{figure}
\centering
\includegraphics{literature_review_files/figure-latex/spacetimegrain-1.pdf}
\caption{\label{fig:spacetimegrain}Not sure about the relevance of this figure}
\end{figure}

\hypertarget{future-directions}{%
\chapter{Future directions}\label{future-directions}}

In order to find general patterns and reliable results about spatio-temporal scaling of biodiversity trends, replications are needed. That is, iteratively computing trends of biodiveristy at certain scales across the globe. However, out of 17 papers, 5 were located in North America and 12 in Europe. This is a striking example of the well-known lack of studies outside Europe and, to a lesser degree, North America \autocite{fraixedas_state_2020}. Yet, biodiversity dynamic across scales in Europe is not representative of global dynamic, especially due to the lesser ecosystem representations, the variations in anthropization, conservation policies etc. Thus, studies replicating the computation of trends at several spatio-temporal scales are needed outside of Europe if one wants to understand the scaling of global biodiversity dynamic. These non-european studies are even more needed when working at spatial grains the size or greater than Europe because in this case, spatial replication means having data about other continents.

Even if a consensus exists about the importance of spatial grains, the final spatial grain of the metric computed is not always specified. One should consider the way the metric is computed, \emph{e.g.} if it was summed, modeled, averaged over the sampling units. According to the method, the spatial grain can vary from the sampling unit to the sampling extent and even more if extrapolation methods are used. This may seem trivial but given the importance of spatial scaling of biodiversity patterns \autocite{storch_untangling_2004}, one has to expect that it will be also important for its dynamic \autocite[\emph{e.g.}][]{chase_species_2019}. By specifying systematically the spatial grain according to the metric used to compute the trend, study of scaling biodiversity trends would be facilitated.

The importance of temporal scaling of biodiversity is known since \textcite{grinnell_role_1922}, who used California birds to demonstrate the species-time relationship, which has since been proven to be common with other bird populations \autocite{white_two-phase_2004}. Thus, as spatial grain, temporal grain is known to be important to explicit. However, there is no consensus on the definition of temporal grain and is thus specified in various ways: sometimes very precised \autocite[\emph{e.g.} time of each census point, as in][]{schipper_contrasting_2016} and sometimes without explicit information \autocite[\emph{e.g.}][\emph{`The sites are visited twice a year (April to early May and late May to June), during which volunteers walk two parallel 1-km-long transect lines {[}\ldots{]}'}]{harrison_assessing_2014}. As for the spatial grain, the temporal grain can vary according to the metric and the way it is computed. The confusion becomes even more important when the trend of the metric is assessed, as the temporal grain of the metric is considered to be representative of the unit time-span of the trend (\emph{i.e.} time-span between \(t\) and \(t+1\)). In other words, when one computes and represent a trend on a figure, he usually use one point either every day, every month or every year. Thus, the temporal grain of the trend is different from the one of the computed metric, especially if one point on the figure embeds several census sessions. If one wants to study the temporal scaling of biodiversity trends, a clear definition of temporal grain for this context needs to be found. From the scientific literature, two definitions emerge:

\textbf{1)} Temporal grain of the trend is the temporal grain of the computed metric. This definition of the temporal grain is similar to the definition of spatial grain. It has the advantage to be very precise. In this case, it should be clearly specified by the authors, according to the method used to compute the metric. However, finding it in the articles is complicated.

\textbf{2)} Temporal grain of the trend is the time-span between two assessment of the metric. This definition makes the temporal grain easier to assess, but with a lack a precision compared to the temporal grain of the sampling plan.

In order to study the temporal scaling of trends of biodiversity, as the first definition is often very complicated to determine in the scientific literature, the second definition can be used by keeping in mind that it represents an approximation.

\hypertarget{conclusion}{%
\chapter{Conclusion}\label{conclusion}}

Reviewing the scientific literature on this topic gives a glimpse of what still needs to be done to better understand the scaling of biodiversity dynamic. One of the first challenge is to find a common definition about spatial and temporal grain when computing the trend. I showed that these definitions are \textbf{1)} metric-related and \textbf{2)} vary according to the way the metric is computed. Whilst spatial grain of a trend is rather intuitive, temporal grain is confusing as it can be considered at two levels: at the metric level or at the temporal trend level. Finally, as birds represent one of the most data furnished taxa of vertebrates, challenges about birds are just a part of what still need to be done for other taxa.

\newpage
\begin{singlespacing}
\printbibliography[heading=bibintoc, title={References}]
\end{singlespacing}

\newpage

\hypertarget{supplementary-materials}{%
\chapter*{Supplementary materials}\label{supplementary-materials}}
\addcontentsline{toc}{chapter}{Supplementary materials}

\begin{landscape}\begingroup\fontsize{10}{12}\selectfont

\begin{longtable}[t]{>{\raggedright\arraybackslash}p{6.5em}>{\raggedright\arraybackslash}p{6.5em}>{\raggedright\arraybackslash}p{6.5em}>{\raggedright\arraybackslash}p{40em}}
\caption{\label{tab:notetable}Supplementary informations about each article}\\
\toprule
Reference & Spatial grain (Km²) & Trend & Note\\
\midrule
\endfirsthead
\caption[]{\label{tab:notetable}Supplementary informations about each article \textit{(continued)}}\\
\toprule
Reference & Spatial grain (Km²) & Trend & Note\\
\midrule
\endhead

\endfoot
\bottomrule
\endlastfoot
\cellcolor{gray!6}{\cite{barnagaud_temporal_2017}} & \cellcolor{gray!6}{Local} & \cellcolor{gray!6}{Increase} & \cellcolor{gray!6}{Not sure that it is at the road scale: "Taxonomic evenness showed a marginal, yet significant, non-linear increase from close to 0.54 in the first decade to 0.56 in the last decade (Table 1), suggesting a light trend towards a more even distribution of species’ abundances among species within local assemblages "}\\
 & Local & Increase & Mean change of SR at the road scales Area of the road = (40/0.8)*(pi*400\^2) with a road of 40 Km with point counts spaced by 0.8 Km and a census radius of 400m\\
\cellcolor{gray!6}{\cite{bowler_geographic_2021}} & \cellcolor{gray!6}{National} & \cellcolor{gray!6}{Stable} & \cellcolor{gray!6}{Metric = MSI, as many and as intense increase (i.e. Czech Rep. and Switzerland) than decrease (i.e. Germany and Denmarl)}\\
\cite{chase_species_2019} & Local & Stable & NA\\
\cellcolor{gray!6}{} & \cellcolor{gray!6}{Regional} & \cellcolor{gray!6}{Stable} & \cellcolor{gray!6}{\vphantom{1} NA}\\
\addlinespace
 & Regional & Stable & NA\\
\cellcolor{gray!6}{} & \cellcolor{gray!6}{Local} & \cellcolor{gray!6}{Increase} & \cellcolor{gray!6}{\vphantom{6} NA}\\
 & Local & Increase & \vphantom{5} NA\\
\cellcolor{gray!6}{\cite{chiron_forecasting_2013}} & \cellcolor{gray!6}{Regional} & \cellcolor{gray!6}{Decrease} & \cellcolor{gray!6}{Concerning the spatial scale, predictions are made using the spatial unit of 4 Km² and the FBI is computed for each region of France, then meanned. Prediction with baseline scenario}\\
 & Regional & Decrease & FBI prediction with CAP greening cenario\\
\addlinespace
\cellcolor{gray!6}{} & \cellcolor{gray!6}{Regional} & \cellcolor{gray!6}{Decrease} & \cellcolor{gray!6}{FBI prediction with No Pillar I scenario}\\
 & Regional & Decrease & FBI prediction with biofuel scenario\\
\cellcolor{gray!6}{\cite{davey_rise_2012}} & \cellcolor{gray!6}{Local} & \cellcolor{gray!6}{Increase} & \cellcolor{gray!6}{Metric = Simpson.They predict the metric using a GAM with spatial resolution of 1 Km². Then they show the trend for the mean value of the metric per year}\\
 & Local & Increase & \vphantom{4} NA\\
\cellcolor{gray!6}{} & \cellcolor{gray!6}{Local} & \cellcolor{gray!6}{Increase} & \cellcolor{gray!6}{\vphantom{3} NA}\\
\addlinespace
\cite{harrison_assessing_2014} & Local & Increase & To assess the metric, they use a GAM to predict the abundance over the entire area of interest (spatial resolution = 1 Km²) and then compute the geometric mean of species abundance = Multi Species Index (as in \cite{studeny_fine-tuning_2013}) from the prediction. Data used to learn the GAM are sampled from plots of 1 Km². Farmland communities\\
\cellcolor{gray!6}{} & \cellcolor{gray!6}{Local} & \cellcolor{gray!6}{Stable} & \cellcolor{gray!6}{Farmland communities, GoF ($\lambda$ = -1) =  weighted towards the rare species}\\
 & Local & Stable & Farmland communities, GoF ( $\lambda$ = -2) weighted towards the common species\\
\cellcolor{gray!6}{\cite{harrison_quantifying_2016}} & \cellcolor{gray!6}{Local} & \cellcolor{gray!6}{Increase} & \cellcolor{gray!6}{Geomteric mean of species abundance, they predict the abundance with resolution of 1 Km² and then computed the metric for each 10000 Km² cell across Great Britain, Visited twice a year}\\
 & Local & Stable & GoF ( $\lambda$ = -1) = toward rare species" The goodness-of-fit-based measure of biodiversity suggests that both rare and common species made gains through much of Britain in the first half of the time period, and losses in the second half.", Visited twice a year / Increase first half and second second halfGoF ( $\lambda$ = -1)\\
\addlinespace
\cellcolor{gray!6}{} & \cellcolor{gray!6}{Local} & \cellcolor{gray!6}{Stable} & \cellcolor{gray!6}{GoF ( $\lambda$ = -2) = toward common species " The goodness-of-fit-based measure of biodiversity suggests that both rare and common species made gains through much of Britain in the first half of the time period, and losses in the second half.", Visited twice a year / Increase first half and second second half}\\
\cite{jarzyna_taxonomic_2018}\cellcolor{gray!6}{} & \cellcolor{gray!6}{Regional} & \cellcolor{gray!6}{Increase} & \cellcolor{gray!6}{\vphantom{4} NA}\\
 & Regional & Increase & \vphantom{3} NA\\
\cellcolor{gray!6}{} & \cellcolor{gray!6}{Regional} & \cellcolor{gray!6}{Increase} & \cellcolor{gray!6}{\vphantom{2} NA}\\
\cellcolor{gray!6}{} & \cellcolor{gray!6}{National} & \cellcolor{gray!6}{Increase} & \cellcolor{gray!6}{\vphantom{1} NA}\\
\addlinespace
 & Global & Decrease & NA\\
 & Regional & Increase & \vphantom{1} NA\\
\cellcolor{gray!6}{} & \cellcolor{gray!6}{Regional} & \cellcolor{gray!6}{Increase} & \cellcolor{gray!6}{NA}\\
 & Regional & Increase & NA\\
 & National & Increase & NA\\
\addlinespace
\cellcolor{gray!6}{} & \cellcolor{gray!6}{Global} & \cellcolor{gray!6}{Stable} & \cellcolor{gray!6}{NA}\\
\cite{pilotto_meta-analysis_2020}\cellcolor{gray!6}{} & \cellcolor{gray!6}{Local} & \cellcolor{gray!6}{Stable} & \cellcolor{gray!6}{"Analyses of the trends in local biodiversity over large spatial scales"}\\
\cellcolor{gray!6}{} & \cellcolor{gray!6}{Local} & \cellcolor{gray!6}{Increase} & \cellcolor{gray!6}{Metric = Simpson, "Analyses of the trends in local biodiversity over large spatial scales"}\\
 & Local & Increase & "Analyses of the trends in local biodiversity over large spatial scales"\\
 & Local & Stable & "Analyses of the trends in local biodiversity over large spatial scales"\\
\addlinespace
\cite{ram_what_2017} & Local & Increase & MSI for forest species, road of 8 Km with no limitations so assumed 200m\\
\cellcolor{gray!6}{} & \cellcolor{gray!6}{Regional} & \cellcolor{gray!6}{Increase} & \cellcolor{gray!6}{SR for forest species meaned over roads, spatial grain = 8* .4 with road of 8 Km and census radius no limitations so assumed 200m}\\
\cite{reif_changes_2013} & Local & Stable & Jaccard index, pairwise comparisions between transects\\
\cellcolor{gray!6}{} & \cellcolor{gray!6}{Local} & \cellcolor{gray!6}{Stable} & \cellcolor{gray!6}{JPSP data, transect scale}\\
 & National & Stable & JPSP data, national scale\\
\addlinespace
\cellcolor{gray!6}{\cite{schipper_contrasting_2016}} & \cellcolor{gray!6}{Local} & \cellcolor{gray!6}{Increase} & \cellcolor{gray!6}{The metric (i.e. geometric mean) is meaned over each road. Area of the road = 50*(pi*400\^2) with 50 census point per road and a census radius of 400m}\\
 & Local & Increase & Metric = Shannon\\
\cellcolor{gray!6}{} & \cellcolor{gray!6}{Local} & \cellcolor{gray!6}{Increase} & \cellcolor{gray!6}{Metric = Simpson}\\
 & Local & Decrease & NA\\
\cellcolor{gray!6}{} & \cellcolor{gray!6}{Local} & \cellcolor{gray!6}{Increase} & \cellcolor{gray!6}{\vphantom{2} NA}\\
\addlinespace
 & Local & Increase & \vphantom{1} NA\\
\cellcolor{gray!6}{} & \cellcolor{gray!6}{Local} & \cellcolor{gray!6}{Increase} & \cellcolor{gray!6}{NA}\\
\cite{sorte_changes_2005} & Local & Decrease & NA\\
\cellcolor{gray!6}{} & \cellcolor{gray!6}{Local} & \cellcolor{gray!6}{Decrease} & \cellcolor{gray!6}{Metric = evenness}\\
 & Local & Increase & The metric is meaned over each road. Area of the road = 50*(pi*400\^2) with 50 census point per road and a census radius of 400m\\
\addlinespace
\cellcolor{gray!6}{\cite{van_turnhout_scale-dependent_2007}} & \cellcolor{gray!6}{Regional} & \cellcolor{gray!6}{Increase} & \cellcolor{gray!6}{For each region, the trend is computed using the mean number of species per atlas square}\\
 & Local & Increase & Mainly increase of SR but the proportion of negative trend were higher than for the regional scale\\
\cellcolor{gray!6}{} & \cellcolor{gray!6}{National} & \cellcolor{gray!6}{Increase} & \cellcolor{gray!6}{National scale}\\
\cite{wretenberg_changes_2010} & Local & Decrease & looking at the trend through different environmental policies, " local species richness (i.e. at the scale of sites = 3 hectares) decreased significantly probably as a result of an overall reduced abundance of several species. "\\
\cellcolor{gray!6}{\cite{inger_common_2015}} & \cellcolor{gray!6}{National} & \cellcolor{gray!6}{Decrease} & \cellcolor{gray!6}{NA}\\
\addlinespace
\cite{donald_agricultural_2001} & National & Decrease & The metric is referred as "Mean population" and the trend is estimated for each european country\\*
\end{longtable}
\endgroup{}
\end{landscape}


%%%%%%%%%%%%%%%% Here is the part that I am using for the bibliography to be displayed in the toc
%%% First step: I define the name and label of the biblio part
%%\chapter{References}\label{references}
%%{
%%% I temporarily redefine the clearpage in order for the bib to not be printed on a new page
%%\renewcommand{\clearpage}{}
%%\printbibliography[heading=none] % I delete the default name of the bib
%%\printbibliography 
%%}
%
\end{document}
