%%%%%%%%%%%%%%%%%%%%%%%%%%%%%%%%%%%%%%%%%%%%%%%%

% Specify the command that you want into the header of the
% index.md file

%%%%%%%%%%%%%%%%%%%%%%%%%%%%%%%%%%%%%%%%%%%%%%%%

% Options for packages loaded elsewhere
\PassOptionsToPackage{unicode}{hyperref}
\PassOptionsToPackage{hyphens}{url}
\PassOptionsToPackage{dvipsnames,svgnames*,x11names*}{xcolor}
%
\documentclass[
  12pt,
  oneside]{report}
%%\usepackage{lmodern}
%
% Set line spacing
\usepackage{setspace}
\setstretch{1.5}

\usepackage{amssymb,amsmath}
\usepackage{ifxetex,ifluatex}
\ifnum 0\ifxetex 1\fi\ifluatex 1\fi=0 % if pdftex
  \usepackage[T1]{fontenc}
  \usepackage[utf8]{inputenc}
  \usepackage{textcomp} % provide euro and other symbols
\else % if luatex or xetex
  \usepackage{unicode-math}
  \defaultfontfeatures{Scale=MatchLowercase}
  \defaultfontfeatures[\rmfamily]{Ligatures=TeX,Scale=1}
\fi
% Use upquote if available, for straight quotes in verbatim environments
\IfFileExists{upquote.sty}{\usepackage{upquote}}{}
\IfFileExists{microtype.sty}{% use microtype if available
  \usepackage[]{microtype}
  \UseMicrotypeSet[protrusion]{basicmath} % disable protrusion for tt fonts
}{}
\makeatletter
\@ifundefined{KOMAClassName}{% if non-KOMA class
  \IfFileExists{parskip.sty}{%
    \usepackage{parskip}
  }{% else
    \setlength{\parindent}{0pt}
    \setlength{\parskip}{6pt plus 2pt minus 1pt}}
}{% if KOMA class
  \KOMAoptions{parskip=half}}
\makeatother
\usepackage{xcolor}
\IfFileExists{xurl.sty}{\usepackage{xurl}}{} % add URL line breaks if available
\IfFileExists{bookmark.sty}{\usepackage{bookmark}}{\usepackage{hyperref}}
\hypersetup{
  pdfauthor={François Leroy, PhD student at CZU},
  colorlinks=true,
  linkcolor=Blue,
  filecolor=Blue,
  citecolor=Blue,
  urlcolor=Blue,
  pdfcreator={LaTeX via pandoc}}
\urlstyle{same} % disable monospaced font for URLs

%% Package geometry
\usepackage[left = 2cm,right = 2cm,top = 2cm,bottom = 2cm]{geometry}
\usepackage{pdflscape}


%\usepackage{longtable} % out of date, now in latex-tools package
\usepackage{booktabs}
% Correct order of tables after \paragraph or \subparagraph
\usepackage{etoolbox}
\makeatletter
\patchcmd\longtable{\par}{\if@noskipsec\mbox{}\fi\par}{}{}
\makeatother
% Allow footnotes in longtable head/foot
\IfFileExists{footnotehyper.sty}{\usepackage{footnotehyper}}{\usepackage{footnote}}
\makesavenoteenv{longtable}
\usepackage{graphicx}
\makeatletter
\def\maxwidth{\ifdim\Gin@nat@width>\linewidth\linewidth\else\Gin@nat@width\fi}
\def\maxheight{\ifdim\Gin@nat@height>\textheight\textheight\else\Gin@nat@height\fi}
\makeatother
% Scale images if necessary, so that they will not overflow the page
% margins by default, and it is still possible to overwrite the defaults
% using explicit options in \includegraphics[width, height, ...]{}
\setkeys{Gin}{width=\maxwidth,height=\maxheight,keepaspectratio}
% Set default figure placement to htbp
\makeatletter
\def\fps@figure{htbp}
\makeatother
\setlength{\emergencystretch}{3em} % prevent overfull lines
\providecommand{\tightlist}{%
  \setlength{\itemsep}{0pt}\setlength{\parskip}{0pt}}
\setcounter{secnumdepth}{5}
%%% Complete the preamble of the LaTeX template
%%%------------------------------------------------------------------------------

%% Bug de bookdown: ne traite plus la déclaration "otherlangs" dans le préambule
% Pour charger les langues, écriture ici en dur du produit de bookdown
% Corrigé le 22/11/2019. A retester régulièrement: supprimer ces lignes si la compilation fonctionne sans elles.
\usepackage{polyglossia}
  \setmainlanguage[variant=american]{english}
  \setotherlanguage[]{french}
% Bug persistant le 28/02/2020

% Advised with polyglossia and babel
\usepackage{csquotes}

% Environnement "Essentiel" en début de chapitre
\usepackage[tikz]{bclogo}
\newenvironment{Essentiel}
  {\begin{bclogo}[logo=\bctrombone, noborder=true, couleur=lightgray!50]{L'essentiel}\parindent0pt}
  {\end{bclogo}}

%% Package fontspec
\usepackage{fontspec}
\setmainfont{calibri}[
  Path           = ./fonts/,
  Extension      = .ttf,
  BoldFont       = calibrib,
  ItalicFont     = calibrili,
  BoldItalicFont = calibriz]

% Rename chapters
% Below, scrpit to prevent the "chapter n" and the space use for it to
% be displayed
\usepackage{titlesec}
\titleformat{\chapter}   
{\Huge}{\thechapter{. }}{0pt}{\Huge}
%{\thechapter{. }}
\titlespacing*{\chapter}{0pt}{-50pt}{10pt}
% -50 is to up the title and 10 is the space with the text below


% When using the natbib biblio package, includes "References" in the table of contents
\usepackage[nottoc]{tocbibind}

% To make the table caption full page width
\usepackage{caption}
\usepackage{booktabs}
\usepackage{longtable}
\usepackage{array}
\usepackage{multirow}
\usepackage{wrapfig}
\usepackage{float}
\usepackage{colortbl}
\usepackage{pdflscape}
\usepackage{tabu}
\usepackage{threeparttable}
\usepackage{threeparttablex}
\usepackage[normalem]{ulem}
\usepackage{makecell}
\usepackage{xcolor}
\ifluatex
  \usepackage{selnolig}  % disable illegal ligatures
\fi
\usepackage[]{natbib}
\bibliographystyle{apa}

\author{François Leroy, PhD student at CZU}
\date{2021-09-14}

% to include pdf
\usepackage{pdfpages}



%%%%%%%%%%%%%%%%%%%%%%%%%%%%%%%%%%%%%%%%%%%%%%%%%%%%%%%%%%%%%
% Start of the documents
\begin{document}

\includepdf[pages = {1}, fitpaper=true]{_assets/coverpage.pdf}

% Roman numbering for content before toc and toc itself
\cleardoublepage 
\pagenumbering{roman}

{
\hypersetup{linkcolor=}
\setcounter{tocdepth}{1}
\tableofcontents
\newpage
}
\vspace{50mm}
\setstretch{1.5}


% Start the arabic numbering at the 1st chapter
\cleardoublepage 
\pagenumbering{arabic}


% The mind, the...
\hypertarget{outline}{%
\chapter*{Outline}\label{outline}}
\addcontentsline{toc}{chapter}{Outline}

Literature review about the link between biodiversity facets trends and spatial/temporal scales.

Make a table of all these papers and \texttt{group\_by(taxa)\ \%\textgreater{}\%\ order\_by(spatial\_scale\ \textbar{}\ temporal\_scale)}. Then see if for each taxa we can find a trend (a bit like in Chase \emph{et al.} 2019 Oikos paper \textbar{} Jarzyna \emph{et al.} 2015 but here I am not making the analysis, just taking the analysis from papers). Best example found so far: \href{https://besjournals.onlinelibrary.wiley.com/doi/10.1111/j.0021-8901.2004.00926.x}{Hill \& Hamer 2004}

\textbf{Meeting with Petr}

\begin{enumerate}
\def\labelenumi{\arabic{enumi})}
\tightlist
\item
  How is the trend of biodiversity metrics linked to spatial grain/spatial extent and temporal grain/temporal extent.
\end{enumerate}

\begin{itemize}
\tightlist
\item
  Talk about papers which does compare (Jarzyna Jetz)
\item
  The inconsistency of reporting scales (especially for models as in Jiguet et al.or Chiron et al., and for MSI metrics)
\end{itemize}

\begin{enumerate}
\def\labelenumi{\arabic{enumi})}
\setcounter{enumi}{1}
\tightlist
\item
  Data heterogeneity
\end{enumerate}

\begin{itemize}
\tightlist
\item
  Lack of spatial replication. That's why this review is important
\item
  The inconsistency of reporting scales (especially for models, and for MSI metrics)
\item
  Lack in other than western countries
\item
  Metric heterogeneity (Freixedas 2001 but they don't mention the macroecological ones as McGill, my struggle because of the time for space substitution as Hill \& Hammer). Some metrics takes into account temporal dynamic but some papers don't look at the trend of these metrics (Jarzyna et al.~2015)
\item
  Can we use space for time substitution as an actual substitution (Hill \& Hammer)
\item
  Pop trends are usually stronger (cf.~all the abundance metrics)
\end{itemize}

\begin{enumerate}
\def\labelenumi{\arabic{enumi})}
\setcounter{enumi}{2}
\tightlist
\item
  Future directions
\end{enumerate}

\begin{itemize}
\tightlist
\item
  Lack of data for other thsn western cvountries: important to harvest data in other countries
\end{itemize}

\hypertarget{introduction}{%
\chapter{Introduction}\label{introduction}}

Human life quality is intrinsically linked to ecosystems state that he is living in. Indeed, ecosystems services extend in a large spectrum of mechanisms including nutrient cycle, food production, or climate and water cycle regulation \citep{pereira_global_2012}. Some of those ecosystem functions are managed by bird biodiversity such as seed dispersal, controls pests or pollinate plant. Unfortunately, anthropogenic stressors like habitat loss, over exploitation, pollution or introduction of invasive species could lead biodiversity to its sixth mass extinction \citep{barnosky_has_2011}.

We now know that the loss of global biodiversity is unprecedented and political decisions has been stated in order to limit it \citep[\emph{e.g.}][2010, 2002]{the_convention_on_biological_diversity_convention_2021}. However, current scientific literature has also shown that temporal trends in local changes of biodiversity can be opposite to trends at larger scales \citep[\emph{e.g.}][]{chase_species_2019}. Thus, current changes in biodiversity is far more complex than a simple global decrease: most of the ecosystems undergo alterations of their communities with changes in species composition \citep{blowes_geography_2019, dornelas_quantifying_2013}. Wonders persist about how the trend of these different metrics of biodiversity are link to the spatial and temporal scales used when measured.

In order to investigate this link between spatio-temporal scales and biodiversity metrics, birds is a relevant taxon. Thanks to the many ornithological monitoring and surveys, we now have a large number of long, high-quality time series on bird populations \citep{bejcek_velke_2016, sauer_north_2013, kamp_population_2021}. Birds are easy to observe, easy to identify and thus many volunteers are motivated to conduct standardized sampling. Given their ability to change quickly of locations, their presence is also a good indicator for ecosystem health and thus several standardized metrics have been created to assess their populations.

However, studying biodiversity can be confusing, especially because several choices must be done. Firstly, the level at which you are looking at the biodiversity must be chosen (\emph{e.g.} taxonomic, functional, phylogenetic diversity). Secondly, one must decide which metric is the most appropriate for his study. There are many facets of biodiversity that can be measured by different metrics depending on the objective of your study. Measures of static biodiversity are commonly used such as species richness or \(\alpha\) diversity \citep[\emph{i.e.} number of species,][]{whittaker_vegetation_1960}, the Shannon index \citep{shannon_mathematical_1948} ,the Simpson index \citep{simpson_measurement_1949} or the Hill number \citep{hill_diversity_1973}. The later three biodiversity indexes take into account the relative abundances of the species and can be considered as the \emph{quality} of the biodiversity. On an other hand, the spatial and temporal \(\beta\) diversity will measure the species turnover and can be measured thanks to Whittaker's \citep{whittaker_evolution_1972}, Sørensen's \citep{sorensen_method_1948} or Jaccard's \citep{jaccard_distribution_1912} dissimilarity indexes \citep[\emph{e.g.}][]{keil_patterns_2012}. All these metrics assess the taxonomic diversity, \emph{i.e.} they use the species as unit. However, it has also been shown that functional and phylogenetic diversity can provide supplementary information on the community structure and its dynamic \citep[\emph{e.g.}][]{mcgill_rebuilding_2006, mouquet_ecophylogenetics_2012, webb_phylogenies_2002}.

An other measure of biodiversity of great interest is the abundance of the populations. As a matter of fact, as individuals react to disturbances of ecosystems by disappearing, the population trends are considered as resulting of the ecosystems health. Overall populations are impossible to assess, but determining the abundance of few indicator species can give a relevant summary on how is going the entire ecosystem \citep{gregory_developing_2005}. These family of metrics are called the multi-species indicators (MSI) and we have seen the emergence of several of them according to the ecosystem considered. We can cite here the farmland bird indicator, woodland bird indicator or one of the most informative, which summarizes these two previous metrics: the Wildland Bird Indicator \citep{gregory_generation_1999, gregory_wild_2010}. The main idea of these metrics is to compute the geometric mean of abundance of few key species over time.

Finally a last class of indicators, not considered here, take into account both species and ecosystems feature into a summarizing index. The most known ones are the Red List Index \citep{butchart_improvements_2007, butchart_using_2005, butchart_measuring_2004} or the Biodiversity Change Index \citep{normander_indicator_2012}.

Here, I propose to review articles assessing the temporal trends of different avian biodiversity metrics and to look at which spatial scales these studies have been done. I decided to consider the most common indicators used to assess biodiversity, such as the diversity indexes (\emph{e.g.} species richness, functional diversity\ldots) or the abundance indexes. Summarizing the trends of these qualitative and/or quantitative avian biodiversity indexes along with their spatial and temporal scales will help to see more clearly how the trends of biodiversity are linked to spatio-temporal scales. It is also important to demonstrate that the information about the sampling plan (\emph{i.e.} spatial scale, time span, temporal scales etc) is not systematically indicated in the scientific literature and can bring confusion to the analysis and comparisons of their trends. I believe that this review can help to have a better overview of the current knowledge on the trend of biodiversity metrics of bird populations.

\textcolor{red}{specify that it is mainly for continental/coastal birds but no look at  islandic communities}

\hypertarget{materials-and-methods}{%
\chapter{Materials and Methods}\label{materials-and-methods}}

For this review, articles of interest were the ones assessing temporal trends of the most common indicators (\emph{i.e.} metrics) of avian biodiversity and specifying spatial and temporal scales. For this, I used the \emph{\enquote{advanced search}} tool of the ISI Web of Science Core collection database with these four following queries:

\begin{enumerate}
\def\labelenumi{\arabic{enumi}.}
\item
  \texttt{AB\ =\ ((biodiversity\ OR\ species\ richness\ OR\ diversity)\ AND\ (temporal\ trend*\ OR\ dynamic*)\ AND\ (bird*\ OR\ avia*))} which resulted in 1346 references.
\item
  \texttt{AB\ =\ ((biodiversity\ change\ index)\ \ AND\ (bird*\ \ OR\ avia*)\ \ AND\ trend*)} which resulted in 60 references.
\item
  \texttt{AB\ =\ ((species\ richness)\ AND\ (bird*\ OR\ avia*)\ AND\ trend*)} which resulted in 313 references.
\item
  \texttt{ALL=(birds\ AND\ species\ richness\ AND\ temporal\ trend)} which resulted in 88 references.
\end{enumerate}

Alternatively, the articles which were often referred to in the relevant articles were also explored.

I decided to take into account only the articles for which there was spatial replicates, \emph{i.e.} where the trend of the metric was assessed at several locations with the same spatial grain (except for the \emph{global scale}, which can not have spatial replicate). With this replications, the trend reported is more reliable. For each query, the title and abstract of the articles were reviewed. When the temporal trend was explicitly specified (either visually or literally), the material and method part was read in order to collect the \emph{spatial grain} of the trend (\emph{i.e.} the area at which the trend is assessed), its \emph{temporal grain} (\emph{i.e.} the time span at which data have been gathered at one census session), the \emph{spatial extent} (\emph{i.e.} the entire area at which the study applies), the \emph{temporal extent} and the \emph{beginning and ending years} of the study as well as the \emph{general trend} of the metric (Tab. \ref{tab:maintable}).

Spatial grain sizes are discretized into four levels: \emph{local} \(<= 25\) \(Km^2\), \emph{regional} \(> 25\) \(Km²\), \emph{national} when an entire country is considered and \emph{global} at the worldwide scale.

Concerning the trend assessment, different papers contain the \emph{p-value} or directly specify the significant trend of the metric. However, a portion of papers gives only visual representations of the trend. For those, the standard error was used when displayed. For the very few only giving the trend, \textcolor{red}{the rule of thumb was applied}. Information can be found in the column \emph{Note} of the Tab. \ref{tab:notetable} of the supplementary material. Moreover, the final trend retained (\emph{i.e.} either \emph{Increase}, \emph{Stable} or \emph{Decrease}) doesn't reflect all the fluctuations of the metric through time but rather the difference between the starting and ending points.

I have found 27 references in which authors were both determining the temporal trend of a metric of interest and explicitly defining the grain size. However, only 15 of them are using spatial replicates and are thus relevant for this study (Tab. \ref{tab:maintable}). The classes of metric are: \emph{Species richness, Evenness, Abundance, Diversity, Temporal beta-diversity, Spatial beta-diversity, Functional diversity, Functional evenness, Functional richness}. Some of this classes contain several different indexes. This is the case for instance for the class \emph{Diversity}, which can be defined by either the Shannon or the Simpson index, or for the class \emph{Abundance} which contains various multi-species indicators.

\hypertarget{results}{%
\chapter{Results}\label{results}}

Articles using spatial replicates to compute the trend of a metric were limited, either due to a lack of data or from an analysis focusing on the global trend of the data. For instance, the Breeding Bird Surveys \citep[\emph{e.g.}][]{sauer_north_2013, kamp_population_2021} follow standardized sampling plan with spatial replications (\emph{i.e.} census plots) but not all the trend of the metrics reported are averaged or summarized at specific grain sizes. As a matter of fact, the common method encountered in the scientific literature is to learn a predictive model from the data, predict the target feature (\emph{i.e.} the variable of interest such as abundance or species richness) and then compute the metric and its trend from the output of the model and at the dataset spatial extent, giving no spatial replicates of the trend \citep[\emph{e.g.}][]{jiguet_modeling_2005, jiguet_french_2012, eglington_disentangling_2012, doxa_low-intensity_2010, sauer_first_2017}.

Overall, there were 9 \emph{Decrease}, 31 \emph{Increase} and 14 \emph{Stable} reliable trends across the literature, without consideration of any metric or spatial scales. Figure \ref{fig:barmetrics} shows the distribution of these trends for each metric. We can see that the species richness and abundance are highly represented.

\begin{figure}
\centering
\includegraphics{literature_review_files/figure-latex/barmetrics-1.pdf}
\caption{\label{fig:barmetrics}Proportion of \emph{Increase}, \emph{Decrease} or \emph{Stable} trends for each of the metric}
\end{figure}

Figure \ref{fig:barspatscale} shows that local scales are more represented than the others and that the number of papers decreases with the increasing spatial scale. This is expected, as the spatial replications get more demanding in organization and resources as the grain size enlarges. It is also clear that the \emph{Increase} of the metrics seems to be dominating at smaller scales. On an other hand, the proportion of \emph{Decrease} is more important at regional scales than at local scales. At national scale, no \emph{Decrease} is witnessed, with almost as \emph{Increase} as \emph{Stable} trends. At the global scale, no \emph{Increase} were found.

\begin{figure}
\centering
\includegraphics{literature_review_files/figure-latex/barspatscale-1.pdf}
\caption{\label{fig:barspatscale}Proportion of \emph{Increase}, \emph{Decrease} or \emph{Stable} trends for each spatial scale}
\end{figure}

Add comment on Figure \ref{fig:barmetricsperspatscale}

\begin{figure}
\centering
\includegraphics{literature_review_files/figure-latex/barmetricsperspatscale-1.pdf}
\caption{\label{fig:barmetricsperspatscale}Proportion of \emph{Increase}, \emph{Decrease} or \emph{Stable} trends for each metric. Each panel represent one spatial scale}
\end{figure}

\hypertarget{discusion}{%
\chapter{Discusion}\label{discusion}}

\begin{enumerate}
\def\labelenumi{\arabic{enumi})}
\setcounter{enumi}{2}
\tightlist
\item
  Future directions
\end{enumerate}

\begin{itemize}
\tightlist
\item
  Lack of data for other than western countries: important to harvest data in other countries. Here, out of the 15 papers, 5 were located in north america and 10 in Europe.
\end{itemize}

\begin{landscape}\begingroup\fontsize{10}{12}\selectfont

\begin{longtable}[t]{>{\raggedright\arraybackslash}p{6.5em}>{\raggedright\arraybackslash}p{6.5em}>{\raggedright\arraybackslash}p{6.5em}>{\raggedleft\arraybackslash}p{6.5em}>{\raggedleft\arraybackslash}p{6.5em}>{\raggedleft\arraybackslash}p{6.5em}>{\raggedright\arraybackslash}p{6.5em}>{\raggedright\arraybackslash}p{6.5em}>{\raggedright\arraybackslash}p{6.5em}}
\caption{\label{tab:maintable}Trends of different metrics of biodiversity at various spatial and temporal scales}\\
\toprule
Reference & Metric & Spatial grain (Km²) & Temporal grain (year) & Spatial extent (Km²) & Temporal extent (year) & Years & Country & Trend\\
\midrule
\endfirsthead
\caption[]{\label{tab:maintable}Trends of different metrics of biodiversity at various spatial and temporal scales \textit{(continued)}}\\
\toprule
Reference & Metric & Spatial grain (Km²) & Temporal grain (year) & Spatial extent (Km²) & Temporal extent (year) & Years & Country & Trend\\
\midrule
\endhead

\endfoot
\bottomrule
\endlastfoot
\cellcolor{gray!6}{\cite{barnagaud_temporal_2017}} & \cellcolor{gray!6}{Evenness} & \cellcolor{gray!6}{Local} & \cellcolor{gray!6}{1.0} & \cellcolor{gray!6}{9834000} & \cellcolor{gray!6}{41} & \cellcolor{gray!6}{1970-2011} & \cellcolor{gray!6}{USA} & \cellcolor{gray!6}{Increase}\\
 & SR & Local & 1.0 & 9834000 & 41 & 1970-2011 & USA & Increase\\
\cellcolor{gray!6}{\cite{bowler_geographic_2021}} & \cellcolor{gray!6}{Abundance} & \cellcolor{gray!6}{National} & \cellcolor{gray!6}{1.0} & \cellcolor{gray!6}{520475} & \cellcolor{gray!6}{27} & \cellcolor{gray!6}{1990-2016} & \cellcolor{gray!6}{Czech Rep., Switzerland, Denmark, Germany} & \cellcolor{gray!6}{Stable}\\
\cite{chase_species_2019} & SR & Local & 5.0 & 2800000 & 30 & 1982–2011 & USA, Canada & Stable\\
\cellcolor{gray!6}{} & \cellcolor{gray!6}{SR} & \cellcolor{gray!6}{Regional} & \cellcolor{gray!6}{5.0} & \cellcolor{gray!6}{2800000} & \cellcolor{gray!6}{30} & \cellcolor{gray!6}{1982–2011} & \cellcolor{gray!6}{USA, Canada} & \cellcolor{gray!6}{\vphantom{1} Stable}\\
\addlinespace
 & SR & Regional & 5.0 & 2800000 & 30 & 1982–2011 & USA, Canada & Stable\\
\cellcolor{gray!6}{} & \cellcolor{gray!6}{SR} & \cellcolor{gray!6}{Local} & \cellcolor{gray!6}{5.0} & \cellcolor{gray!6}{2800000} & \cellcolor{gray!6}{30} & \cellcolor{gray!6}{1982–2011} & \cellcolor{gray!6}{USA, Canada} & \cellcolor{gray!6}{\vphantom{1} Increase}\\
 & SR & Local & 5.0 & 2800000 & 30 & 1982–2011 & USA, Canada & Increase\\
\cite{chiron_forecasting_2013} & Abundance & Regional & 1.0 & 643801 & 14 & 2007-2020 & France & \vphantom{2} Decrease\\
\cellcolor{gray!6}{} & \cellcolor{gray!6}{Abundance} & \cellcolor{gray!6}{Regional} & \cellcolor{gray!6}{1.0} & \cellcolor{gray!6}{643801} & \cellcolor{gray!6}{14} & \cellcolor{gray!6}{2007-2020} & \cellcolor{gray!6}{France} & \cellcolor{gray!6}{\vphantom{1} Decrease}\\
\addlinespace
 & Abundance & Regional & 1.0 & 643801 & 14 & 2007-2020 & France & Decrease\\
 & Abundance & Regional & 1.0 & 643801 & 14 & 2007-2020 & France & Decrease\\
\cellcolor{gray!6}{\cite{davey_rise_2012}} & \cellcolor{gray!6}{Diversity} & \cellcolor{gray!6}{Local} & \cellcolor{gray!6}{1.0} & \cellcolor{gray!6}{242495} & \cellcolor{gray!6}{13} & \cellcolor{gray!6}{1994-2006} & \cellcolor{gray!6}{UK} & \cellcolor{gray!6}{Increase}\\
 & Evenness & Local & 1.0 & 242495 & 13 & 1994-2006 & UK & Increase\\
\cellcolor{gray!6}{} & \cellcolor{gray!6}{SR} & \cellcolor{gray!6}{Local} & \cellcolor{gray!6}{1.0} & \cellcolor{gray!6}{242495} & \cellcolor{gray!6}{13} & \cellcolor{gray!6}{1994-2006} & \cellcolor{gray!6}{UK} & \cellcolor{gray!6}{Increase}\\
\addlinespace
\cite{harrison_assessing_2014} & Abundance & Local & 1.0 & 200000 & 18 & 1994-2011 & Great Britain, UK & Increase\\
\cellcolor{gray!6}{} & \cellcolor{gray!6}{Abundance} & \cellcolor{gray!6}{Local} & \cellcolor{gray!6}{1.0} & \cellcolor{gray!6}{200000} & \cellcolor{gray!6}{18} & \cellcolor{gray!6}{1994-2011} & \cellcolor{gray!6}{Great Britain, UK} & \cellcolor{gray!6}{\vphantom{1} Stable}\\
 & Abundance & Local & 1.0 & 200000 & 18 & 1994-2011 & Great Britain, UK & Stable\\
\cellcolor{gray!6}{\cite{harrison_quantifying_2016}} & \cellcolor{gray!6}{Abundance} & \cellcolor{gray!6}{Local} & \cellcolor{gray!6}{0.5} & \cellcolor{gray!6}{NA} & \cellcolor{gray!6}{20} & \cellcolor{gray!6}{1994-2013} & \cellcolor{gray!6}{UK} & \cellcolor{gray!6}{Increase}\\
 & Abundance & Local & 0.5 & NA & 20 & 1994-2013 & UK & \vphantom{1} Stable\\
\addlinespace
\cellcolor{gray!6}{} & \cellcolor{gray!6}{Abundance} & \cellcolor{gray!6}{Local} & \cellcolor{gray!6}{0.5} & \cellcolor{gray!6}{NA} & \cellcolor{gray!6}{20} & \cellcolor{gray!6}{1994-2013} & \cellcolor{gray!6}{UK} & \cellcolor{gray!6}{Stable}\\
\cite{jarzyna_taxonomic_2018}\cellcolor{gray!6}{} & \cellcolor{gray!6}{SR} & \cellcolor{gray!6}{Regional} & \cellcolor{gray!6}{1.0} & \cellcolor{gray!6}{9834000} & \cellcolor{gray!6}{45} & \cellcolor{gray!6}{1969-2013} & \cellcolor{gray!6}{USA} & \cellcolor{gray!6}{\vphantom{1} Increase}\\
 & SR & Regional & 1.0 & 9834000 & 45 & 1969-2013 & USA & Increase\\
 & SR & Regional & 1.0 & 9834000 & 45 & 1969-2013 & USA & Increase\\
\cellcolor{gray!6}{} & \cellcolor{gray!6}{SR} & \cellcolor{gray!6}{National} & \cellcolor{gray!6}{1.0} & \cellcolor{gray!6}{9834000} & \cellcolor{gray!6}{45} & \cellcolor{gray!6}{1969-2013} & \cellcolor{gray!6}{USA} & \cellcolor{gray!6}{Increase}\\
\addlinespace
 & SR & Global & 1.0 & 148940000 & 45 & 1969-2013 & World & Decrease\\
\cellcolor{gray!6}{} & \cellcolor{gray!6}{Temporal beta-diversity} & \cellcolor{gray!6}{Regional} & \cellcolor{gray!6}{1.0} & \cellcolor{gray!6}{9834000} & \cellcolor{gray!6}{45} & \cellcolor{gray!6}{1969-2013} & \cellcolor{gray!6}{USA} & \cellcolor{gray!6}{\vphantom{2} Increase}\\
 & Temporal beta-diversity & Regional & 1.0 & 9834000 & 45 & 1969-2013 & USA & \vphantom{1} Increase\\
\cellcolor{gray!6}{} & \cellcolor{gray!6}{Temporal beta-diversity} & \cellcolor{gray!6}{Regional} & \cellcolor{gray!6}{1.0} & \cellcolor{gray!6}{9834000} & \cellcolor{gray!6}{45} & \cellcolor{gray!6}{1969-2013} & \cellcolor{gray!6}{USA} & \cellcolor{gray!6}{Increase}\\
 & Temporal beta-diversity & National & 1.0 & 9834000 & 45 & 1969-2013 & USA & Increase\\
\addlinespace
\cellcolor{gray!6}{} & \cellcolor{gray!6}{Temporal beta-diversity} & \cellcolor{gray!6}{Global} & \cellcolor{gray!6}{1.0} & \cellcolor{gray!6}{148940000} & \cellcolor{gray!6}{45} & \cellcolor{gray!6}{1969-2013} & \cellcolor{gray!6}{World} & \cellcolor{gray!6}{Stable}\\
\cite{pilotto_meta-analysis_2020} & Abundance & Local & NA & 10180000 & NA & NA & Europe & Stable\\
\cellcolor{gray!6}{} & \cellcolor{gray!6}{Diversity} & \cellcolor{gray!6}{Local} & \cellcolor{gray!6}{NA} & \cellcolor{gray!6}{10180000} & \cellcolor{gray!6}{NA} & \cellcolor{gray!6}{NA} & \cellcolor{gray!6}{Europe} & \cellcolor{gray!6}{Increase}\\
 & SR & Local & NA & 10180000 & NA & NA & Europe & Increase\\
\cellcolor{gray!6}{} & \cellcolor{gray!6}{Temporal beta-diversity} & \cellcolor{gray!6}{Local} & \cellcolor{gray!6}{NA} & \cellcolor{gray!6}{10180000} & \cellcolor{gray!6}{NA} & \cellcolor{gray!6}{NA} & \cellcolor{gray!6}{Europe} & \cellcolor{gray!6}{Stable}\\
\addlinespace
\cite{ram_what_2017} & Abundance & Local & 1.0 & 350000 & 18 & 1998-2015 & Sweden & Increase\\
\cellcolor{gray!6}{} & \cellcolor{gray!6}{SR} & \cellcolor{gray!6}{Regional} & \cellcolor{gray!6}{1.0} & \cellcolor{gray!6}{350000} & \cellcolor{gray!6}{18} & \cellcolor{gray!6}{1998-2015} & \cellcolor{gray!6}{Sweden} & \cellcolor{gray!6}{Increase}\\
\cite{reif_changes_2013} & Spatial beta-diversity & Local & 1.0 & 79000 & 23 & 1982-2004 & Czech Rep. & Stable\\
\cellcolor{gray!6}{} & \cellcolor{gray!6}{SR} & \cellcolor{gray!6}{Local} & \cellcolor{gray!6}{1.0} & \cellcolor{gray!6}{79000} & \cellcolor{gray!6}{23} & \cellcolor{gray!6}{1982-2004} & \cellcolor{gray!6}{Czech Rep.} & \cellcolor{gray!6}{Stable}\\
 & SR & National & 1.0 & 79000 & 23 & 1982-2004 & Czech Rep. & Stable\\
\addlinespace
\cellcolor{gray!6}{\cite{schipper_contrasting_2016}} & \cellcolor{gray!6}{Abundance} & \cellcolor{gray!6}{Local} & \cellcolor{gray!6}{5.0} & \cellcolor{gray!6}{24710000} & \cellcolor{gray!6}{40} & \cellcolor{gray!6}{1971-2010} & \cellcolor{gray!6}{Canada, USA, Mexico} & \cellcolor{gray!6}{Increase}\\
 & Diversity & Local & 5.0 & 24710000 & 40 & 1971-2010 & Canada, USA, Mexico & \vphantom{1} Increase\\
\cellcolor{gray!6}{} & \cellcolor{gray!6}{Diversity} & \cellcolor{gray!6}{Local} & \cellcolor{gray!6}{5.0} & \cellcolor{gray!6}{24710000} & \cellcolor{gray!6}{40} & \cellcolor{gray!6}{1971-2010} & \cellcolor{gray!6}{Canada, USA, Mexico} & \cellcolor{gray!6}{Increase}\\
 & Functional diversity & Local & 5.0 & 24710000 & 40 & 1971-2010 & Canada, USA, Mexico & Decrease\\
\cellcolor{gray!6}{} & \cellcolor{gray!6}{Functional evenness} & \cellcolor{gray!6}{Local} & \cellcolor{gray!6}{5.0} & \cellcolor{gray!6}{24710000} & \cellcolor{gray!6}{40} & \cellcolor{gray!6}{1971-2010} & \cellcolor{gray!6}{Canada, USA, Mexico} & \cellcolor{gray!6}{Increase}\\
\addlinespace
 & Functional richness & Local & 5.0 & 24710000 & 40 & 1971-2010 & Canada, USA, Mexico & Increase\\
\cellcolor{gray!6}{} & \cellcolor{gray!6}{SR} & \cellcolor{gray!6}{Local} & \cellcolor{gray!6}{5.0} & \cellcolor{gray!6}{24710000} & \cellcolor{gray!6}{40} & \cellcolor{gray!6}{1971-2010} & \cellcolor{gray!6}{Canada, USA, Mexico} & \cellcolor{gray!6}{Increase}\\
\cite{sorte_changes_2005} & Abundance & Local & 1.0 & 9834000 & 36 & 1968-2003 & USA & Decrease\\
\cellcolor{gray!6}{} & \cellcolor{gray!6}{Evenness} & \cellcolor{gray!6}{Local} & \cellcolor{gray!6}{1.0} & \cellcolor{gray!6}{9834000} & \cellcolor{gray!6}{36} & \cellcolor{gray!6}{-} & \cellcolor{gray!6}{USA} & \cellcolor{gray!6}{Decrease}\\
 & SR & Local & 1.0 & 9834000 & 36 & 1968-2003 & USA & Increase\\
\addlinespace
\cellcolor{gray!6}{\cite{van_turnhout_scale-dependent_2007}} & \cellcolor{gray!6}{SR} & \cellcolor{gray!6}{Regional} & \cellcolor{gray!6}{4.0} & \cellcolor{gray!6}{41543} & \cellcolor{gray!6}{28} & \cellcolor{gray!6}{1973-2000} & \cellcolor{gray!6}{Netherlands} & \cellcolor{gray!6}{Increase}\\
 & SR & Local & 4.0 & 41543 & 28 & 1973-2000 & Netherlands & Increase\\
\cellcolor{gray!6}{} & \cellcolor{gray!6}{SR} & \cellcolor{gray!6}{National} & \cellcolor{gray!6}{4.0} & \cellcolor{gray!6}{41543} & \cellcolor{gray!6}{28} & \cellcolor{gray!6}{1973-2000} & \cellcolor{gray!6}{Netherlands} & \cellcolor{gray!6}{Increase}\\
\cite{wretenberg_changes_2010} & SR & Local & 1.0 & 1800 & 11 & 1994-2004 & Sweden & Decrease\\*
\end{longtable}
\endgroup{}
\end{landscape}

\hypertarget{supplementary-materials}{%
\chapter*{Supplementary materials}\label{supplementary-materials}}
\addcontentsline{toc}{chapter}{Supplementary materials}

\begin{landscape}\begingroup\fontsize{10}{12}\selectfont

\begin{longtable}[t]{>{\raggedright\arraybackslash}p{6.5em}>{\raggedright\arraybackslash}p{6.5em}>{\raggedright\arraybackslash}p{6.5em}>{\raggedright\arraybackslash}p{40em}}
\caption{\label{tab:notetable}Supplementary informations about each article}\\
\toprule
Reference & Spatial grain (Km²) & Trend & Note\\
\midrule
\endfirsthead
\caption[]{\label{tab:notetable}Supplementary informations about each article \textit{(continued)}}\\
\toprule
Reference & Spatial grain (Km²) & Trend & Note\\
\midrule
\endhead

\endfoot
\bottomrule
\endlastfoot
\cellcolor{gray!6}{\cite{barnagaud_temporal_2017}} & \cellcolor{gray!6}{Local} & \cellcolor{gray!6}{Increase} & \cellcolor{gray!6}{Not sure that it is at the road scale: "Taxonomic evenness showed a marginal, yet significant, non-linear increase from close to 0.54 in the first decade to 0.56 in the last decade (Table 1), suggesting a light trend towards a more even distribution of species’ abundances among species within local assemblages "}\\
 & Local & Increase & Mean change of SR at the road scales Area of the road = (40/0.8)*(pi*400\^2) with a road of 40 Km with point counts spaced by 0.8 Km and a census radius of 400m\\
\cellcolor{gray!6}{\cite{bowler_geographic_2021}} & \cellcolor{gray!6}{National} & \cellcolor{gray!6}{Stable} & \cellcolor{gray!6}{Metric = MSI, as many and as intense increase (i.e. Czech Rep. and Switzerland) than decrease (i.e. Germany and Denmarl)}\\
\cite{chase_species_2019} & Local & Stable & NA\\
\cellcolor{gray!6}{} & \cellcolor{gray!6}{Regional} & \cellcolor{gray!6}{Stable} & \cellcolor{gray!6}{\vphantom{1} NA}\\
\addlinespace
 & Regional & Stable & NA\\
\cellcolor{gray!6}{} & \cellcolor{gray!6}{Local} & \cellcolor{gray!6}{Increase} & \cellcolor{gray!6}{\vphantom{6} NA}\\
 & Local & Increase & \vphantom{5} NA\\
\cellcolor{gray!6}{\cite{chiron_forecasting_2013}} & \cellcolor{gray!6}{Regional} & \cellcolor{gray!6}{Decrease} & \cellcolor{gray!6}{Concerning the spatial scale, predictions are made using the spatial unit of 4 Km² and the FBI is computed for each region of France, then meanned. Prediction with baseline scenario}\\
 & Regional & Decrease & FBI prediction with CAP greening cenario\\
\addlinespace
\cellcolor{gray!6}{} & \cellcolor{gray!6}{Regional} & \cellcolor{gray!6}{Decrease} & \cellcolor{gray!6}{FBI prediction with No Pillar I scenario}\\
 & Regional & Decrease & FBI prediction with biofuel scenario\\
\cellcolor{gray!6}{\cite{davey_rise_2012}} & \cellcolor{gray!6}{Local} & \cellcolor{gray!6}{Increase} & \cellcolor{gray!6}{Metric = Simpson.They predict the metric using a GAM with spatial resolution of 1 Km². Then they show the trend for the mean value of the metric per year}\\
 & Local & Increase & \vphantom{4} NA\\
\cellcolor{gray!6}{} & \cellcolor{gray!6}{Local} & \cellcolor{gray!6}{Increase} & \cellcolor{gray!6}{\vphantom{3} NA}\\
\addlinespace
\cite{harrison_assessing_2014} & Local & Increase & To assess the metric, they use a GAM to predict the abundance over the entire area of interest (spatial resolution = 1 Km²) and then compute the geometric mean of species abundance = Multi Species Index (as in \cite{studeny_fine-tuning_2013}) from the prediction. Data used to learn the GAM are sampled from plots of 1 Km². Farmland communities\\
\cellcolor{gray!6}{} & \cellcolor{gray!6}{Local} & \cellcolor{gray!6}{Stable} & \cellcolor{gray!6}{Farmland communities, GoF ($\lambda$ = -1) =  weighted towards the rare species}\\
 & Local & Stable & Farmland communities, GoF ( $\lambda$ = -2) weighted towards the common species\\
\cellcolor{gray!6}{\cite{harrison_quantifying_2016}} & \cellcolor{gray!6}{Local} & \cellcolor{gray!6}{Increase} & \cellcolor{gray!6}{Geomteric mean of species abundance, they predict the abundance with resolution of 1 Km² and then computed the metric for each 10000 Km² cell across Great Britain, Visited twice a year}\\
 & Local & Stable & GoF ( $\lambda$ = -1) = toward rare species" The goodness-of-fit-based measure of biodiversity suggests that both rare and common species made gains through much of Britain in the first half of the time period, and losses in the second half.", Visited twice a year / Increase first half and second second halfGoF ( $\lambda$ = -1)\\
\addlinespace
\cellcolor{gray!6}{} & \cellcolor{gray!6}{Local} & \cellcolor{gray!6}{Stable} & \cellcolor{gray!6}{GoF ( $\lambda$ = -2) = toward common species " The goodness-of-fit-based measure of biodiversity suggests that both rare and common species made gains through much of Britain in the first half of the time period, and losses in the second half.", Visited twice a year / Increase first half and second second half}\\
\cite{jarzyna_taxonomic_2018}\cellcolor{gray!6}{} & \cellcolor{gray!6}{Regional} & \cellcolor{gray!6}{Increase} & \cellcolor{gray!6}{\vphantom{4} NA}\\
 & Regional & Increase & \vphantom{3} NA\\
\cellcolor{gray!6}{} & \cellcolor{gray!6}{Regional} & \cellcolor{gray!6}{Increase} & \cellcolor{gray!6}{\vphantom{2} NA}\\
\cellcolor{gray!6}{} & \cellcolor{gray!6}{National} & \cellcolor{gray!6}{Increase} & \cellcolor{gray!6}{\vphantom{1} NA}\\
\addlinespace
 & Global & Decrease & NA\\
 & Regional & Increase & \vphantom{1} NA\\
\cellcolor{gray!6}{} & \cellcolor{gray!6}{Regional} & \cellcolor{gray!6}{Increase} & \cellcolor{gray!6}{NA}\\
 & Regional & Increase & NA\\
 & National & Increase & NA\\
\addlinespace
\cellcolor{gray!6}{} & \cellcolor{gray!6}{Global} & \cellcolor{gray!6}{Stable} & \cellcolor{gray!6}{NA}\\
\cite{pilotto_meta-analysis_2020}\cellcolor{gray!6}{} & \cellcolor{gray!6}{Local} & \cellcolor{gray!6}{Stable} & \cellcolor{gray!6}{"Analyses of the trends in local biodiversity over large spatial scales"}\\
\cellcolor{gray!6}{} & \cellcolor{gray!6}{Local} & \cellcolor{gray!6}{Increase} & \cellcolor{gray!6}{Metric = Simpson, "Analyses of the trends in local biodiversity over large spatial scales"}\\
 & Local & Increase & "Analyses of the trends in local biodiversity over large spatial scales"\\
 & Local & Stable & "Analyses of the trends in local biodiversity over large spatial scales"\\
\addlinespace
\cite{ram_what_2017} & Local & Increase & MSI for forest species, road of 8 Km with no limitations so assumed 200m\\
\cellcolor{gray!6}{} & \cellcolor{gray!6}{Regional} & \cellcolor{gray!6}{Increase} & \cellcolor{gray!6}{SR for forest species meaned over roads, spatial grain = 8* .4 with road of 8 Km and census radius no limitations so assumed 200m}\\
\cite{reif_changes_2013} & Local & Stable & Jaccard index, pairwise comparisions between transects\\
\cellcolor{gray!6}{} & \cellcolor{gray!6}{Local} & \cellcolor{gray!6}{Stable} & \cellcolor{gray!6}{JPSP data, transect scale}\\
 & National & Stable & JPSP data, national scale\\
\addlinespace
\cellcolor{gray!6}{\cite{schipper_contrasting_2016}} & \cellcolor{gray!6}{Local} & \cellcolor{gray!6}{Increase} & \cellcolor{gray!6}{The metric (i.e. geometric mean) is meaned over each road. Area of the road = 50*(pi*400\^2) with 50 census point per road and a census radius of 400m}\\
 & Local & Increase & Metric = Shannon\\
\cellcolor{gray!6}{} & \cellcolor{gray!6}{Local} & \cellcolor{gray!6}{Increase} & \cellcolor{gray!6}{Metric = Simpson}\\
 & Local & Decrease & NA\\
\cellcolor{gray!6}{} & \cellcolor{gray!6}{Local} & \cellcolor{gray!6}{Increase} & \cellcolor{gray!6}{\vphantom{2} NA}\\
\addlinespace
 & Local & Increase & \vphantom{1} NA\\
\cellcolor{gray!6}{} & \cellcolor{gray!6}{Local} & \cellcolor{gray!6}{Increase} & \cellcolor{gray!6}{NA}\\
\cite{sorte_changes_2005} & Local & Decrease & NA\\
\cellcolor{gray!6}{} & \cellcolor{gray!6}{Local} & \cellcolor{gray!6}{Decrease} & \cellcolor{gray!6}{Metric = evenness}\\
 & Local & Increase & The metric is meaned over each road. Area of the road = 50*(pi*400\^2) with 50 census point per road and a census radius of 400m\\
\addlinespace
\cellcolor{gray!6}{\cite{van_turnhout_scale-dependent_2007}} & \cellcolor{gray!6}{Regional} & \cellcolor{gray!6}{Increase} & \cellcolor{gray!6}{For each region, the trend is computed}\\
 & Local & Increase & Mainly increase of SR but the proportion of negative trend were higher than for the regional scale\\
\cellcolor{gray!6}{} & \cellcolor{gray!6}{National} & \cellcolor{gray!6}{Increase} & \cellcolor{gray!6}{National scale}\\
\cite{wretenberg_changes_2010} & Local & Decrease & looking at the trend through different environmental policies, " local species richness (i.e. at the scale of sites = 3 hectares) decreased significantly probably as a result of an overall reduced abundance of several species. "\\*
\end{longtable}
\endgroup{}
\end{landscape}

\singlespacing


\renewcommand\bibname{References}
  \bibliography{references.bib}

\end{document}
